\subsection{Purpose, Approach, Goal}
\paragraph{My purpose} for writing this document is to further the discussion of the VSIPL data model to see if we can come up with a common agreement for a generalized VSIPL data model which has enough flexibility for use with any language which one might want to write a VSIPL library for.
\paragraph{My approach} will be to discuss what I consider the \cvl{} data model to be and add information and insight from what I have learned writing \pyjv{}.  
\paragraph{Comment}
I have never had much luck with UML either reading it or producing it. I think it may be much overblown as a methodology for specifying systems.  In spite of that, I will attempt some basic diagrams. I do not know UML so people who know UML, assuming such people exist, need to complain and correct if they think it is needed.
\\[6pt]
I have an example for a Strassen matrix product routine from Golub and Van Loan Matrix Computations algorithm 1.3.1.  I previously implemented that in C++ VSIPL.  I do not know if that example is current with the current VSIPL++ library and I have not been able to install a working library from the current distribution on my systems.  In any case at one time it worked and should still be a valid example for my purposes.  I also have a run-able example for \cvl{} and I used the \cvl{} example as a template for a \pyjv{} example.  So for the current 3 languages of interest I have a common example function that I will use to discuss the data model.
\\[6pt]
There will be some discussion paragraphs where I talk about things I know little about, like the data model for C++ VSIPL, and ask questions or make comments designed to (hopefully) foster some discussion and illumination from the C++ \ttbf{Implementors} and \ttbf{Users}.
  
\paragraph{My goal} is to move the discussion and try to get a few people involved who may be interested in the topic.  If we want to move the spec we need some consensus. If I don't get much interest I will move off and just do my own thing.
