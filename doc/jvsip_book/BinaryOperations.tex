\subsubsection*{Binary Operations\hspace*{\fill}\hyperlink{ElementwiseOperations}{(up)}\hypertarget{binaryOperations}{}}\addcontentsline{toc}{subsubsection}{Binary Operations}
Elementwise functions requiring two inputs, either two \ttbf{view}s or a \ttbf{view} and a scalar, are called binary operations.\\
Note that the table in this document is somewhat shorter than the table in the C VSIPL document. For this table, for instance, an \ttbf{add} is only broken out as one function. For C VSIPL there are three function for add depending on the argument list shapes. I decided to avoid that here, partly because for \pyjv I can overload the call and a single method or function name is satisfactory.\\
\begin{table}[H]
\caption{Binary Operations}
\label{tab:binaryOperations}
\begin{center}
\begin{tabular}{|l|l|}\hline
\hlnkFunc{add} & Add\\
\hlnkFunc{div} & Divide\\
\hlnkFunc{expoavg} & Exponential Average\\
\hlnkFunc{hypot} & Hypotenuse\\
\hlnkFunc{jmul} & Conjugate Multiply\\
\hlnkFunc{mul} & Multiply\\
\hlnkFunc{vmmul} & Vector Matrix Multiply\\
\hlnkFunc{sub} & Subtract\\
\hline\end{tabular}
\end{center}
\label{default}
\end{table}%
