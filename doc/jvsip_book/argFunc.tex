\afunc{arg}{Compute the radian value argument of complex elements in the interval $[-\pi,\pi]$. An Unary Operation. See table \ref{tab:unaryOperations}}
\cvsiplh
\newline \hspace*{.8cm} \vspace*{.1cm} \textbf{Available Functions }
\newline \hspace*{1.1cm} {
\ttfamily
\begin{tabular}[H]{l}
vsip\_scalar\_d vsip\_arg\_d(vsip\_cscalar\_d);\\
vsip\_scalar\_f vsip\_arg\_f(vsip\_cscalar\_f);\\
void vsip\_marg\_d(const vsip\_cmview\_d*, const vsip\_mview\_d* );\\
void vsip\_marg\_f(const vsip\_cmview\_f*, const vsip\_mview\_f* );\\
void vsip\_varg\_d(const vsip\_cvview\_d*,const vsip\_vview\_d*);\\
void vsip\_varg\_f(const vsip\_cvview\_f*, const vsip\_vview\_f* );\\
\end{tabular}
}
\pyjvsiph
\viewmthd{yes}{yes}{No}{out=in.arg}
\apyfunc{yes}{out = arg(in,out)}
\newline \hspace*{.8cm}\textbf{Comment}\\
\hspace*{.8cm}\parbox{11cm}{\vspace*{.2cm}
\begin{itemize}
\item{Since \ttbf{arg} takes a view of \emph{depth} complex and outputs to a view of \emph{depth} real of the same \emph{shape} and \emph{precision} as the input view the \ttbf{arg} \pyjvmethod will create a view of the proper type and size and return it.}
\item{The \ttbf{arg} \pyjvfunc works the same as the C VSIPL function except a convenience pointer is returned to the output view}
\item{For the \pyjvfunc limited in-place functionality exists with replacement of the real or imaginary view of the input with the output. For instance \ilCode{out=arg(in,in.realview)} works fine.}
\end{itemize}}
