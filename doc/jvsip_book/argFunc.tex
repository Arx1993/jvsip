\afuncT{arg}{Compute the radian value argument of complex elements in the interval $[-\pi,\pi]$. An Unary Operation.}{unaryOperations}
\\\cvsiplh
\begin{cfuncs}
vsip\_scalar\_d vsip\_arg\_d(vsip\_cscalar\_d);\Bs\\
vsip\_scalar\_f vsip\_arg\_f(vsip\_cscalar\_f);\Bs\\
void vsip\_marg\_d(const~vsip\_cmview\_d*, const~vsip\_mview\_d*);\Bs\\
void vsip\_marg\_f(const~vsip\_cmview\_f*, const~vsip\_mview\_f*);\Bs\\
void vsip\_varg\_d(const~vsip\_cvview\_d*, const~vsip\_vview\_d*);\Bs\\
void vsip\_varg\_f(const~vsip\_cvview\_f*, const~vsip\_vview\_f*);\Bs\\
\end{cfuncs}
\pyjvsiph
\viewmthd{yes}{yes}{No}{out=in.arg}
\apyfunc{yes}{out = arg(in,out)}
\begin{comments}
\item{The \ttbf{arg} method takes a view of \emph{depth} complex and outputs to a view of \emph{depth} real of the same \emph{shape} and \emph{precision} as the input view.
\subitem The \ttbf{arg} method will create a view of the proper type and size and return it.}
\item{The \ttbf{arg} function works the same as the C VSIPL function except a convenience pointer is returned to the output view}
\item{For the function limited in-place functionality exists with replacement of the real or imaginary view of the input with the output. For instance \ilCode{out=arg(in,in.realview)} works fine.}
\end{comments}