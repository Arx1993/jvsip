\afuncT{ma}{Multiply and add. An element-wise function.}{ternaryOperations}
\\\cvsiplh
\\ \hspace*{.8cm} \vspace*{.1cm} \textbf{Available Functions }
\\ \hspace*{1.1cm} {
\ttfamily
\begin{tabular}[H]{l}
void vsip\_cvma\_d(const vsip\_cvview\_d*, const vsip\_cvview\_d*,\\*\hspace{.7cm}const vsip\_cvview\_d*, const vsip\_cvview\_d*);\\
void vsip\_cvma\_f(const vsip\_cvview\_f*, const vsip\_cvview\_f*,\\*\hspace{.7cm}const vsip\_cvview\_f*, const vsip\_cvview\_f*);\\
void vsip\_cvsma\_d(const vsip\_cvview\_d*, vsip\_cscalar\_d,\\*\hspace{.7cm}const vsip\_cvview\_d*, const vsip\_cvview\_d*);\\
void vsip\_cvsma\_f(const vsip\_cvview\_f*, vsip\_cscalar\_f,\\*\hspace{.7cm}const vsip\_cvview\_f*, const vsip\_cvview\_f*);\\
void vsip\_vma\_d(const vsip\_vview\_d*, const vsip\_vview\_d*,\\*\hspace{.7cm}const vsip\_vview\_d*, const vsip\_vview\_d*);\\
void vsip\_vma\_f(const vsip\_vview\_f*, const vsip\_vview\_f*,\\*\hspace{.7cm}const vsip\_vview\_f*, const vsip\_vview\_f*);\\
void vsip\_vsma\_d(const vsip\_vview\_d*, vsip\_scalar\_d,\\*\hspace{.7cm}const vsip\_vview\_d*, const vsip\_vview\_d*);\\
void vsip\_vsma\_f(const vsip\_vview\_f*, vsip\_scalar\_f,\\*\hspace{.7cm}const vsip\_vview\_f*, const vsip\_vview\_f*);\\
void vsip\_cvsmsa\_d(const vsip\_cvview\_d*, vsip\_cscalar\_d,\\*\hspace{.7cm}vsip\_cscalar\_d, const vsip\_cvview\_d*);\\
void vsip\_cvsmsa\_f(const vsip\_cvview\_f*, vsip\_cscalar\_f,\\*\hspace{.7cm}vsip\_cscalar\_f, const vsip\_cvview\_f*);\\
void vsip\_vsmsa\_d(const vsip\_vview\_d*, vsip\_scalar\_d,\\*\hspace{.7cm}vsip\_scalar\_d, const vsip\_vview\_d*);\\
void vsip\_vsmsa\_f(const vsip\_vview\_f*, vsip\_scalar\_f,\\*\hspace{.7cm}vsip\_scalar\_f, const vsip\_vview\_f*);\\
\end{tabular}
}
\pyComment{\item{The C VSIPL spec has separate man pages for multiply-add functions containing scalar arguments, and those containing only \ttbf{view} arguments.}}
\\\pyjvsiph
\viewmthd{No}{NA}{NA}{NA}
\apyfunc{yes}{\ttbf{out = ma(in1,in2,in3,out)}}
\pyComment{\item{Argument \ttbf{in1} is always a \ttbf{view}, argument \ttbf{in2} is either a \ttbf{view} or a scalar and argument \ttbf{in3} is either a \ttbf{view} or a scalar.}
\item{The \ttbf{ma} function works much the same as the C VSIPL version except that a convenience pointer to the output \ttbf{view} is returned.}
\item{This may be done in-place if an input \ttbf{view} is the same as the output \ttbf{view}.}}