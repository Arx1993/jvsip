\afuncT{asin}{Inverse Cosine. An elementary math function.}{elementaryMath}
\\\cvsiplh
\begin{cfuncs}
vsip\_scalar\_f vsip\_asin\_f(vsip\_scalar\_f a);\Bs\\
vsip\_scalar\_d vsip\_asin\_d(vsip\_scalar\_d a);\Bs\\
void vsip\_masin\_d(const~vsip\_mview\_d*, const~vsip\_mview\_d*);\Bs\\
void vsip\_masin\_f(const~vsip\_mview\_f*, const~vsip\_mview\_f*);\Bs\\
void vsip\_vasin\_d(const~vsip\_vview\_d*, const~vsip\_vview\_d*);\Bs\\
void vsip\_vasin\_f(const~vsip\_vview\_f*, const~vsip\_vview\_f*);\Bs\\
\end{cfuncs}
\pyjvsiph
\viewmthd{yes}{yes}{yes}{inOut.asin}
\apyfunc{yes}{out = asin(in,out)}
\pyComment{
\item{The \ttbf{asin} function works much the same as the C VSIPL version except that a convenience pointer to the output view is returned. This may be done in-place if \ttbf{in==out}.}}
