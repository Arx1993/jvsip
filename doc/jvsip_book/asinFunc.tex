\afunc{asin}{Inverse Cosine. An elementary math function. See table \ref{tab:elementaryMath}.}
\cvsiplh
\newline \hspace*{.8cm} \vspace*{.1cm} \textbf{Available Functions }
\newline \hspace*{1.1cm} {
\ttfamily
\begin{tabular}[H]{l}
vsip\_scalar\_f vsip\_asin\_f(vsip\_scalar\_f a);\\
vsip\_scalar\_d vsip\_asin\_d(vsip\_scalar\_d a);\\
void vsip\_masin\_d(\\*
\hspace{1cm}const vsip\_mview\_d*, const vsip\_mview\_d*);\\
void vsip\_masin\_f(\\*
\hspace{1cm}const vsip\_mview\_f*, const vsip\_mview\_f*);\\
void vsip\_vasin\_d(\\*
\hspace{1cm}const vsip\_vview\_d*, const vsip\_vview\_d*);\\
void vsip\_vasin\_f(\\*
\hspace{1cm}const vsip\_vview\_f*, const vsip\_vview\_f*);\\
\end{tabular}
}
\pyjvsiph
\viewmthd{yes}{yes}{yes}{inOut.asin}
\apyfunc{yes}{out = asin(in,out)}
\pyComment{\item{The \ttbf{asin} function works much the same as the C VSIPL version except that a convenience pointer to the output view is returned. This may be done in-place if \ttbf{in==out}.}}
