\clearpage
\hypertarget{blockFunc}{\large \textbf{Block Function Set}}\vspace{.2cm}\\
\\\cvsiplh 
\newline \hspace*{.8cm} \vspace*{.1cm} \textbf{Available Functions }
%
\newline \hspace*{0.5cm} {
\texttt{
\begin{tabular}[H]{l}
\multicolumn{1}{c}{\rmfamily \bfseries Create Block Object\vspace{.1cm}}\\\hline
vsip\_block\_bl* vsip\_blockcreate\_bl(\\*\hspace{1cm}size\_t, vsip\_memory\_hint);\\
vsip\_block\_d* vsip\_blockcreate\_d(\\*\hspace{1cm}size\_t, vsip\_memory\_hint);\\
vsip\_block\_f* vsip\_blockcreate\_f(\\*\hspace{1cm}size\_t, vsip\_memory\_hint);\\
vsip\_block\_i* vsip\_blockcreate\_i(\\*\hspace{1cm}size\_t, vsip\_memory\_hint);\\
vsip\_block\_mi* vsip\_blockcreate\_mi(\\*\hspace{1cm}size\_t, vsip\_memory\_hint);\\
vsip\_block\_si* vsip\_blockcreate\_si(\\*\hspace{1cm}size\_t, vsip\_memory\_hint);\\
vsip\_block\_uc* vsip\_blockcreate\_uc(\\*\hspace{1cm}size\_t, vsip\_memory\_hint);\\
vsip\_block\_vi* vsip\_blockcreate\_vi(\\*\hspace{1cm}size\_t, vsip\_memory\_hint);\\
vsip\_cblock\_d* vsip\_cblockcreate\_d(\\*\hspace{1cm}size\_t, vsip\_memory\_hint);\\
vsip\_cblock\_f* vsip\_cblockcreate\_f(\\*\hspace{1cm}size\_t, vsip\_memory\_hint);\vspace{.2cm}\\ 
\hline\hline\\
\multicolumn{1}{c}{\rmfamily \bfseries Free Block Object\vspace{.1cm}}\\\hline
void vsip\_blockdestroy\_bl(vsip\_block\_bl*);\\
void vsip\_blockdestroy\_d(vsip\_block\_d*);\\
void vsip\_blockdestroy\_f(vsip\_block\_f*);\\
void vsip\_blockdestroy\_i(vsip\_block\_i*);\\
void vsip\_blockdestroy\_mi(vsip\_block\_mi*);\\
void vsip\_blockdestroy\_si(vsip\_block\_si*);\\
void vsip\_blockdestroy\_uc(vsip\_block\_uc*);\\
void vsip\_blockdestroy\_vi(vsip\_block\_vi *);\\
void vsip\_cblockdestroy\_d(vsip\_cblock\_d*);\\
void vsip\_cblockdestroy\_f(vsip\_cblock\_f*);\\
\end{tabular}
}}
\\\pyjvsiph
\\ \hspace*{.8cm}\parbox{.9\textwidth}{For \pyjv{} a block class has been defined. Methods which expand on the functionality of \cvl{} have been defined.  The \ttbf{Block} object will contain a reference to a \cvl{} block. This block is freed when the \pyjv{} block is destroyed.}
\\[6pt] \hspace*{.8cm}{\textbf{View Methods\vspace{.2cm}}\\
\hspace*{1.1cm}\parbox{.9\textwidth}{
No special view methods exist for the block class but a derived block will be created if a real or imaginary view is extracted from a view associated with a complex block.}
\\[6pt]
%%
\hspace*{1.1cm}\textbf{Example: }\vspace*{.1cm}\\
%
\hspace*{.8cm}{\textbf{Block Class Methods\vspace*{.2cm}}\\
\hspace*{1.cm}\parbox{.9\textwidth}{To create an \ttbf{Block} object do\\
\hspace*{1.cm}\ttbf{aBlock = Block(t,length)} \\
Where \ttbf{t} is a string indicating the type of \ttbf{Block} object to create and \ttbf{length} is the number of scalar values the \ttbf{Block} object will store.}
\\
\begin{table}
\caption{Types for Block Creation}
\begin{center}\begin{tabular}{|l l|}
%%
\multicolumn{2}{c}{\Ts\parbox[t]{.6\textwidth}{\center{\rmfamily \bfseries Block Types}}}\Bs\\\hline
'block\_d' & Real \ttbf{block}; double precision \Bs\\\hline
'block\_f' & Real \ttbf{block}; float precision\Bs\\\hline
'cblock\_d' & Complex \ttbf{block}; double precision\Bs\\\hline
'cblock\_f' & Complex \ttbf{block}; float precision\Bs\\\hline
%%%
\end{tabular}
\end{center}
\end{table}