\afuncT{block}{Block Function Set}{BlockObjects}
%\hypertarget{blockFunc}{\large \textbf{Block Function Set}}\vspace{.2cm}\\
\\\cvsiplh 
\begin{cfuncs}
\multicolumn{1}{c}{\Ts\rmfamily \bfseries Create Block Object\Bs}\\\hline
vsip\_block\_bl* vsip\_blockcreate\_bl(size\_t, vsip\_memory\_hint);\Bs\\
vsip\_block\_d* vsip\_blockcreate\_d(size\_t, vsip\_memory\_hint);\Bs\\
vsip\_block\_f* vsip\_blockcreate\_f(size\_t, vsip\_memory\_hint);\Bs\\
vsip\_block\_i* vsip\_blockcreate\_i(size\_t, vsip\_memory\_hint);\Bs\\
vsip\_block\_mi* vsip\_blockcreate\_mi(size\_t, vsip\_memory\_hint);\Bs\\
vsip\_block\_si* vsip\_blockcreate\_si(size\_t, vsip\_memory\_hint);\Bs\\
vsip\_block\_uc* vsip\_blockcreate\_uc(size\_t, vsip\_memory\_hint);\Bs\\
vsip\_block\_vi* vsip\_blockcreate\_vi(size\_t, vsip\_memory\_hint);\Bs\\
vsip\_cblock\_d* vsip\_cblockcreate\_d(size\_t, vsip\_memory\_hint);\Bs\\
vsip\_cblock\_f* vsip\_cblockcreate\_f(size\_t, vsip\_memory\_hint);\Bs\\ 
\hline\hline\\
\multicolumn{1}{c}{\Ts\rmfamily \bfseries Free Block Object\Bs}\\\hline
void vsip\_blockdestroy\_bl(vsip\_block\_bl*);\Bs\\
void vsip\_blockdestroy\_d(vsip\_block\_d*);\Bs\\
void vsip\_blockdestroy\_f(vsip\_block\_f*);\Bs\\
void vsip\_blockdestroy\_i(vsip\_block\_i*);\Bs\\
void vsip\_blockdestroy\_mi(vsip\_block\_mi*);\Bs\\
void vsip\_blockdestroy\_si(vsip\_block\_si*);\Bs\\
void vsip\_blockdestroy\_uc(vsip\_block\_uc*);\Bs\\
void vsip\_blockdestroy\_vi(vsip\_block\_vi *);\Bs\\
void vsip\_cblockdestroy\_d(vsip\_cblock\_d*);\Bs\\
void vsip\_cblockdestroy\_f(vsip\_cblock\_f*);\Bs\\
\end{cfuncs}
\pyjvsiph
\\\hspace*{1cm}\parbox{.9\textwidth}{For \pyjv{} a block class has been defined. For comments below assume we have an instantiated block called \ttbf{aBlock}.}
\begin{itemize}[leftmargin=1.35cm]
\item Methods which expand on the functionality of {\cvl{}} have been defined.
\item In \cvl{} there is no method to recover a blocks length after it has been created. In \pyjv{} \ttbf{block} class we define two methods to recover a blocks length.
    \begin{itemize}
    \item{Blocks support the \ttbf{len} python function so the number of elements a block can hold (the blocks length) may be found with \ilCode{len(aBlock)}.}
    \item{We can also get a blocks length using a property as in \ilCode{aBlock.length}.}
     \end{itemize}
\item The data stored in a \ttbf{block} is accessed through a \ttbf{view} object.
    \begin{itemize}
    \item The \ttbf{bind} method is the general way to create and bind a \ttbf{view} to a block as in\\* \ilCode{aMatrix = aBlock.bind(offset,colstride,collength,rowstride,rowlength)}\\* or \\*\ilCode{aVector = aBlock.bind(offset,stride,length)}.
    \item There is a convenience method to return a vector \ttbf{view} of the block elements as in\\* \ilCode{aVector = aBlock.vector}.\\* This is equivalent to\\* \ilCode{aVector = aBlock.bind(0,1,aBlock.length)}.
    \end{itemize}
%
\item The \ttbf{Block} object will contain a reference to a \cvl{} block.
    \begin{itemize} 
    \item To recover the \cvl{} block object use the \ilCode{vsip} property as in \\*
    \ilCode{aCVSIPLblock = aBlock.vsip}
    \item The \cvl{} block contained in the \pyjv{} block object is freed when the \pyjv{} block is destroyed by the python garbage collector.
    \end{itemize}
\item A block will create a new block of the same size, depth and precision with the \ilCode{empty} property as in\\*
\ilCode{aNewBlock = aBlock.empty}\\*
There is no equivalent \cvl{} function.
\end{itemize}
%
\Ts\hspace*{.8cm}{\textbf{Derived Block\vspace{.2cm}}\\
\hspace*{1.1cm}\parbox{.9\textwidth}{
A derived block will be created if a real or imaginary view is extracted from a view associated with a complex block. You can recover the derived block from the new view and use it like any other block. There is no function to create a derived block directly.}
\\
%%
\hspace*{1.1cm}\textbf{Example: }\vspace*{.1cm}\\
\hspace*{1.1cm}\parbox{.9\textwidth}{\ilCode{aRealView=aComplexView.realview} or \ilCode{aRealView=aComplexView.imagview}\\*
\ilCode{aDerivedBlock=aRealView.block}\Ts}
\hspace*{.8cm}{\Ts\textbf{Block Class\vspace*{.2cm}\Bs}\\
\hspace*{1.cm}\parbox{.9\textwidth}{To create an \ttbf{Block} object do\\
\hspace*{1.cm}\ttbf{aBlock = Block(t,length)} \\
Where \ttbf{t} is a string indicating the type of \ttbf{Block} object to create and \ttbf{length} is the number of scalar values the \ttbf{Block} object will store.}
\\
\begin{table}[h!]
\caption{Types for Block Creation}
\begin{center}
\begin{tabular}{|l|l|}\hline
'block\_d' & Real \ttbf{block}; double precision \Bs\\\hline
'block\_f' & Real \ttbf{block}; float precision\Bs\\\hline
'cblock\_d' & Complex \ttbf{block}; double precision\Bs\\\hline
'cblock\_f' & Complex \ttbf{block}; float precision\Bs\\\hline
%%%
\end{tabular}
\end{center}
\end{table}