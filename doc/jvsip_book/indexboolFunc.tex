\afuncT{indexbool}{Index a boolean. Given a boolean \ttbf{view} fill an index vector of indices of true values and return the number of true values found.}{selectionOperations}
\\\cvsiplh
\afh
{
\ttfamily
\\\hspace*{.04\textwidth}\begin{tabular}[H]{l}
vsip\_length vsip\_mindexbool(const vsip\_mview\_bl*, vsip\_vview\_mi*);\\
vsip\_length vsip\_vindexbool(const vsip\_vview\_bl*, vsip\_vview\_vi*);\\
\end{tabular}
}
\\\pyjvsiph
%
\viewmthd{Yes}{Yes}{No}{\ttbf{aIndex=aBoolView.indexbool}}
%
\apyfunc{Yes}{\ttbf{nTrue=indexbool(aBoolView,aIndexView)}}
%
\pyComment{
\item{The \ttbf{indexbool} function works much the same as the \cvl{} function and returns the number of true values.}
\item{The \ttbf{indexbool} \ttbf{view} method will create the index vector and return it. The number of true values returned is the length of the returned index vector.}
}