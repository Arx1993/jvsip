\subsubsection*{Window Functions\hspace*{\fill}\hyperlink{SignalProcessing}{(up)}\hypertarget{windowFunctions}{}} \addcontentsline{toc}{subsubsection}{Window Functions}
When windows were defined in the VSIPL specification they were defined as standalone functions to create a compact block with a vector \ttbf{view} and fill the view with the window coefficients. I think this was an unfortunate way to do it; we would have been better off to first create the \ttbf{view} and then call a function to fill the view with the coefficients.

The method of window creation in C VSIPL makes it difficult to encapsulate windows into the \pyjv methods and functions; and I don't want to create a special class just for windows. Consequently window creation has become part of the class for \pyjv.
\begin{table}[H]
\caption{Window Functions}
\label{tab:windowFunctions}
\begin{center}
\begin{tabular}{|l|l|} \hline
\hlnkFunc{blackman} & Blackman Window\\
\hlnkFunc{cheby} & Chebyshev Window\\
\hlnkFunc{hanning} & Hanning Window\\
\hlnkFunc{kaiser} & Kaiser of Window\\
\hline\end{tabular}
\end{center}
\label{default}
\end{table}%