\afuncT{atan2}{Arctangent of Two Arguments; An elementwise function. Computes the four quadrant radian value, $[-\pi,\pi]$, of the arctangent of the ratio of the corresponding elements of two input views.}{elementaryMath}
\\\cvsiplh
\begin{cfuncs}
vsip\_scalar\_d vsip\_atan2\_d(vsip\_scalar\_d, vsip\_scalar\_d);\Bs\\
vsip\_scalar\_f vsip\_atan2\_f(vsip\_scalar\_f, vsip\_scalar\_f);\Bs\\
void vsip\_matan2\_d(const~vsip\_mview\_d*, const~vsip\_mview\_d*, const~vsip\_mview\_d*);\Bs\\
void vsip\_matan2\_f(const~vsip\_mview\_f*, const~vsip\_mview\_f*, const~vsip\_mview\_f*);\Bs\\
void vsip\_vatan2\_d(const~vsip\_vview\_d*, const~vsip\_vview\_d*, const~vsip\_vview\_d*);\Bs\\
void vsip\_vatan2\_f(const~vsip\_vview\_f*, const~vsip\_vview\_f*, const~vsip\_vview\_f*);\Bs\\
\end{cfuncs}
\pyjvsiph
\viewmthd{no}{NA}{NA}{NA}
\apyfunc{yes}{out = atan2(inOne,inTwo,out)}
\pyComment{
\item{The \ttbf{atan2} function works much the same as the C VSIPL version except that a convenience pointer to the output view is returned. This may be done in-place if \ttbf{inOne==out} or \ttbf{inTwo==out}.}}