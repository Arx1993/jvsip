\documentclass[10pt,oneside,a4paper]{book}
\usepackage[margin=2cm]{geometry}
%\usepackage{showframe}
\usepackage{graphicx}
\usepackage{parskip}
\usepackage{blindtext}
\usepackage[linktoc=all]{hyperref}
\setcounter{tocdepth}{3}
\usepackage{hyperref}
\hypersetup{colorlinks=true}
\usepackage{mathtools}
\usepackage[T1]{fontenc} 
\usepackage{lmodern}
\usepackage{listings} 
\usepackage{float}
\usepackage{makeidx}
\usepackage{minted}
\usepackage[hypcap]{caption}
\newcounter{cexctr} \setcounter{cexctr}{1}
\newcounter{pyjvexctr} \setcounter{pyjvexctr}{1}
\newcommand{\ttbf}[1]{{\ttfamily \bfseries #1}}
\newcommand{\ilCode}[1]{{\small {\ttfamily \bfseries #1}}}
\newcommand{\afuncT}[3]{\clearpage{\Large \textbf{\hypertarget{{\ttfamily \bfseries#1Func}}{#1}}{\hspace*{\fill}\hypertarget{#1Func}{}\hyperlink{#3}{up}}} \vspace{.01\textheight}\\ \hspace*{.02\textwidth} \parbox{.97\textwidth}{#2 } \vspace{.005\textheight}}
%hlinkFunk used for hyperlink of table name with function page.
\newcommand{\hlnkFunc}[1]{\hyperlink{#1Func}{\texttt{#1}}}
\newcommand{\hlnkFuncT}[2]{\hyperlink{#1Func}{\texttt{#2}}}
%
\newcommand{\cvsiplh}{ \hspace*{.02\textwidth} {\large \textbf{C VSIPL}}\vspace{.005\textheight}}
%
\newcommand{\pyjvsiph}{\hspace*{.02\textwidth} {\large \textbf{pyJvsip}}\vspace*{.005\textheight}}
%
\newcommand{\pyjvComment}[1]{\newline \hspace*{.3cm} \textbf{Comments}\\ \hspace*{.6cm}\parbox{10.8cm}{ \vspace*{.1cm}\begin{itemize}{ #1}\end{itemize}}}
%
\newcommand{\pyComment}[1]{\newline \hspace*{.8cm} \textbf{Comments}\\ \hspace*{.8cm}\parbox{.9\textwidth}{\vspace*{.1cm}\begin{itemize}{ #1} \end{itemize}}}
%
\newcommand{\viewmthd}[4]{\\\hspace*{.04\textwidth}\textbf{View Method}
\\ \hspace*{.06\textwidth}\textbf{Available: }#1\hspace{.25cm}\textbf{Property: }#2\hspace{.25cm}\textbf{In-Place: } #3 
\\ \hspace*{.06\textwidth}\textbf{Example:} {\ttfamily #4}}
%
\newcommand{\apyfunc}[2]{\newline \hspace*{.04\textwidth}\textbf{Function} 
\\ \hspace*{.06\textwidth}\textbf{Available:} #1 
\\ \hspace*{.06\textwidth}\textbf{Example:} {\ttfamily #2}}
%use as \pyJvsip{}
\newcommand{\pyjv}{{\ttbf{pyJvsip}}}
%use as \jv{}
\newcommand{\jv}{\ttbf{JVSIP}}
%use as \cvl{}
\newcommand{\cvl}{\ttbf{C VSIPL}}
% from http://mirrors.rit.edu/CTAN/info/digests/ttn/ttn2n3.pdf
\newcommand\Ts{\rule{0pt}{2.6ex}}
\newcommand\Bs{\rule[-1.2ex]{0pt}{0pt}}
\makeindex
\begin{document}
\afuncT{sin}{Sine; An element-wise function. Input \ttbf{view} elements are assumed to be in radians.}{elementaryMath}
\\\cvsiplh
\newline \hspace*{.8cm} \vspace*{.1cm} \textbf{Available Functions }
\newline \hspace*{1.1cm} {
\ttfamily
\begin{tabular}[H]{l}
vsip\_scalar\_f vsip\_sin\_f(vsip\_scalar\_f a);\\
vsip\_scalar\_d vsip\_sin\_d(vsip\_scalar\_d a);\\
void vsip\_msin\_d(\\*
\hspace{1cm}const vsip\_mview\_d*, const vsip\_mview\_d*);\\
void vsip\_msin\_f(\\*
\hspace{1cm}const vsip\_mview\_f*, const vsip\_mview\_f*);\\
void vsip\_vsin\_d(\\*
\hspace{1cm}const vsip\_vview\_d*, const vsip\_vview\_d*);\\
void vsip\_vsin\_f(\\*
\hspace{1cm}const vsip\_vview\_f*, const vsip\_vview\_f*);\\
\end{tabular}
}
\\\pyjvsiph
\viewmthd{yes}{yes}{yes}{inOut.sin}
\apyfunc{yes}{out = sin(in,out)}
\newline\hspace*{1.2cm}\parbox{10.8cm}{\vspace*{.1cm}The \ttbf{sin} function works much the same as the C VSIPL version except that a convenience pointer to the output view is returned. This may be done in-place if \ttbf{in==out}.}

\end{document}