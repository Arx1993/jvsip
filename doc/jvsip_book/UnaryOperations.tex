\subsubsection*{Unary Operations\hspace*{\fill}\hyperlink{ElementwiseOperations}{(up)}\hypertarget{unaryOperations}{}}\addcontentsline{toc}{subsubsection}{Unary Operations}
Unary operations involve calculations on a single \ttbf{view}. Functions which involve a calculation where the answer is a scalar, such as \ttbf{sumval} generally have a \ttbf{val} as part of the root name. 
\begin{table}[H]
\caption{Unary Operations}
\label{tab:unaryOperations}
\begin{center}
\begin{tabular}{|l|l|}\hline
\hlnkFunc{arg} & Argument\\
\hlnkFunc{ceil} & Ceiling\\
\hlnkFunc{conj} & Conjugate\\
\hlnkFunc{cumsum} & Cumulative Sum\\
\hlnkFunc{euler} & Euler\\
\hlnkFunc{floor} & Floor\\
\hlnkFunc{mag} & Magnitude\\
\hlnkFunc{cmagsq} & Complex Magnitude Squared\\
\hlnkFunc{meanval} & Mean Value\\
\hlnkFunc{meansqval} & Mean Square Value\\
\hlnkFunc{modulate} & Modulate\\
\hlnkFunc{neg} & Negate\\
\hlnkFunc{recip} & Reciprocal\\
\hlnkFunc{round} & Round\\
\hlnkFunc{rsqrt} & reciprocal Square Root\\
\hlnkFunc{sq} & Square\\
\hlnkFunc{sumval} & Sum Value\\
\hlnkFunc{sumsqval} & Sum of Squares Value\\
\hline\end{tabular}
\end{center}
%\label{default}
\end{table}%
