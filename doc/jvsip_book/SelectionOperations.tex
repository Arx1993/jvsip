\subsubsection*{Selection Operations}\addcontentsline{toc}{subsubsection}{Selection Operations}
Selection operations involve some logical comparison and, based upon the result, an answer is \emph{selected} and returned; either as a scalar output (signified by \ttbf{val} ending the root name), or elementwise into an appropriately sized output \ttbf{view}. 
\begin{table}[H]
\caption{Selection Operations}
\label{tab:selectionOperations}
\begin{center}
\begin{tabular}{|l|l|}\hline
\hlnkFunc{clip} & Clip\\
\hlnkFunc{first} & Find First Vector Index\\
\hlnkFunc{invclip} & Inverted Clip\\
\hlnkFunc{indexbool} & Index a Boolean \ttbf{view}\\
\hlnkFunc{max} & Maximum By-Element between \ttbf{view}s\\
\hlnkFunc{maxmg} & Maximum Magnitude By-Element between \ttbf{view}s\\
\hlnkFunc{cmaxmgsq} & Maximum Magnitude Squared By-Element between complex \ttbf{view}s\\
\hlnkFunc{cmaxmgsqval} & Maximum Magnitude Squared Value of a complex \ttbf{view}\\
\hlnkFunc{maxmgval} & Maximum Magnitude Value of a \ttbf{view}\\
\hlnkFunc{maxval} & Maximum Value in a \ttbf{view}\\
\hlnkFunc{min} & Minimum Elementwise between \ttbf{view}s\\
\hlnkFunc{minmg} & Minimum Magnitude By-Element between \ttbf{view}s\\
\hlnkFunc{cminmgsq} & Minimum Magnitude Squared By-Element between complex \ttbf{view}s\\
\hlnkFunc{cminmgsqval} &  Minimum Magnitude Squared Value of a complex \ttbf{view}\\
\hlnkFunc{minmgval} & Minimum Magnitude Value of a \ttbf{view}\\
\hlnkFunc{minval} & Minimum Value in a \ttbf{view}\\
\hline\end{tabular}
\end{center}
\label{default}
\end{table}%
