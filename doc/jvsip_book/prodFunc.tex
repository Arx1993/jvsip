\afuncT{prod}{Matrix product.}{matrixOperations}
\\\cvsiplh
\\ \hspace*{.8cm} \vspace*{.1cm} \textbf{Available Functions }
\\ \hspace*{0.03\textwidth} {
\ttfamily
\begin{tabular}[H]{l}
void vsip\_cmprod\_d(\\*\hspace{.6cm}const vsip\_cmview\_d*, const vsip\_cmview\_d*, const vsip\_cmview\_d*);\\ 
void vsip\_cmprod\_f(\\*\hspace{.6cm}const vsip\_cmview\_f*, const vsip\_cmview\_f*, const vsip\_cmview\_f*);\\ 
void vsip\_cmvprod\_d(\\*\hspace{.6cm}const vsip\_cmview\_d*, const vsip\_cvview\_d*, const vsip\_cvview\_d*);\\ 
void vsip\_cmvprod\_f(\\*\hspace{.6cm}const vsip\_cmview\_f*, const vsip\_cvview\_f*, const vsip\_cvview\_f*);\\ 
void vsip\_cvmprod\_d(\\*\hspace{.6cm}const vsip\_cvview\_d*, const vsip\_cmview\_d*, const vsip\_cvview\_d*);\\ 
void vsip\_cvmprod\_f(\\*\hspace{.6cm}const vsip\_cvview\_f*, const vsip\_cmview\_f*, const vsip\_cvview\_f*);\\ 
void vsip\_mprod\_d(\\*\hspace{.6cm}const vsip\_mview\_d*, const vsip\_mview\_d*, const vsip\_mview\_d*);\\ 
void vsip\_mprod\_f(\\*\hspace{.6cm}const vsip\_mview\_f*, const vsip\_mview\_f*, const vsip\_mview\_f*);\\ 
void vsip\_mvprod\_d(\\*\hspace{.6cm}const vsip\_mview\_d*, const vsip\_vview\_d*, const vsip\_vview\_d*);\\ 
void vsip\_mvprod\_f(\\*\hspace{.6cm}const vsip\_mview\_f*, const vsip\_vview\_f*, const vsip\_vview\_f*);\\ 
void vsip\_vmprod\_d(\\*\hspace{.6cm}const vsip\_vview\_d*, const vsip\_mview\_d*, const vsip\_vview\_d*);\\ 
void vsip\_vmprod\_f(\\*\hspace{.6cm}const vsip\_vview\_f*, const vsip\_mview\_f*, const vsip\_vview\_f*);\\ 
\end{tabular}
}
\\\pyjvsiph
\viewmthd{yes}{No}{No}{out=inOne.prod(inTwo)}
\apyfunc{yes}{out = prod(inOne,inTwo,out)}
\pyComment{
\item{The \ttbf{prod} function works much the same as the C VSIPL version except that a
 convenience pointer to the output view is returned. This may not be done in-place.}
}
