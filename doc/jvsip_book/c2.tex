\chapter{Functions}
\section*{Introduction}
In this chapter I give basic usage information for the functions included in the JVSIP implementation of the C VSIPL specification and also related information for the \pyjv python module.  There are many functions so I may miss a few.

Usage information may also be found by reading the C VSIPL specification, either the old one included with the JVSIP distribution or the newer one developed by the HPEC working group of the OMG.  I currently recommend sticking with the old one included with the JVSIP distribution.  There is a lot of information about C VSIPL in the specification so C VSIPL information in this document will not be extensive; and since \pyjv has no specification document I will spend more time covering the \pyjv methodology.

I try and include information on the \pyjv methods and functions collocated with the corresponding C VSIPL information.  Reading the pyJvsip.py module file is also encouraged.  \ttbf{PyJvsip} includes some functionality not (directly) part of C VSIPL.  I will try and highlight these special cases.  

For python information the python help mechanism has also been supported somewhat; but keeping that information correct, up-to-date, and available for every function is a work in progress. 

Keep in mind this chapters main purpose is as a go-to reference for proper incantations when writing code. Except for the introductory sections it is probably not something you will want to read.

In order to have some reasonable ordering of the functions the alphabetical listing is based upon a root function name, not the actual vsip function. For instance the second function in the list is the \ilCode{add} function. There are several \ilCode{add} functions in the Core profile. All of them are placed together under \ilCode{add}.

When a C VSIPL function requires a special object it needs support functions to create the object, and destroy it, and perhaps query it for its attributes. For instance to do a discrete Fourier transform one needs a function to create an FFT object, a function to do the actual FFT using the FFT object, and a function to destroy the FFT object when it is no longer needed. The author calls functions which are designed to work together to do a single job function sets. Function sets are placed together under a single heading. For instance all the functions involved with doing an FFT are placed under the FFT heading.

As discussed in chapter one python supports polymorphism, and object oriented programing. A \pyjv object is an instantiation of a python class definition. The python object will contain a C VSIPL object as an instance variable as well as other information needed by \pyjv. For this reason the python garbage collector will destroy C VSIPL objects when no reference to the \pyjv object exists.

Because of the true object oriented nature of \pyjv there are methods defined for every class which accomplish most of the functionality of C VSIPL. \ttbf{PyJvsip} also defines many functions which operate on the \pyjv objects. Frequently you can use either a method or a function. This information is reflected in the JVSIP function list.

No attempt is made to be exhaustive in the function descriptions. Those interested in more detail are directed to the VSIPL specification document included with the JVSIP distribution. In addition various examples included in this document will provide more detail on the use of some of the more complicated functions.

\section*{C VSIPL Specification}
The main document on which \ttbf{JVSIP} is based is the \emph{VSIPL 1.3 API} as approved by the VSIPL Forum on January 31, 2008.  That document is included with the \ttbf{JVSIP} distribution.  The purpose of this section is to provide a roadmap for people who are familiar with this C VSIPL specification to get around in this \ttbf{JVSIP} manual.  Here I provide tables in an order matching the \emph{VSIPL 1.3 API} specification with links to the same information as presented  in the \ttbf{JVSIP} manual.
\subsection*{Summary of VSIPL Types}
\subsection*{Support Functions}
\subsection*{Scalar Functions}
In general I do not define scalar functions in \pyjv.  Ease of use is a major goal of the \pyjv module and to further this goal I decided to scalars used by or returned by \pyjv functions should be normal python scalars. Using scalar functions (such as $\cos$, $\sin$, etc.) imported from the math module or the numpy module should work fine. That said, you can always use the C VSIPL scalar functions directly since they are in the \ttbf{vsip} module which is included in the \pyjv module.
\subsection*{Random Number Generation}
\subsection*{Vector \& Elementwise Operations}
Many of the functions are simple operations which are done on each element in a matrix or vector. The tables referenced here list these with a link to the corresponding function page.
    \subsubsection*{Elementary Math}\addcontentsline{toc}{subsubsection}{Elementary Math}
Elementary math functions constitute elementwise applications of elementary operations on \ttbf{view}s. The term \emph{elementary} is somewhat arbitrary but includes trigonometric functions, log functions, and exponential functions. Functions here (for elements) are defined by C 89 in the \ttbf{math.h} header file. \ttbf{JVSIP} generally uses this math library to do the calculations for these functions.
\begin{table}[H]
\caption{Elementary Math Functions\ref{tab:elementwiseChapter}}
\label{tab:elementaryMath}
\begin{center}
\begin{tabular}{|l|l|}
\hlnkFunc{acos} & Arccosine\\
\hlnkFunc{asin} & Arcsine\\
\hlnkFunc{atan} & Arctangent\\
\hlnkFunc{atan2} & Arctangent of Two Arguments\\
\hlnkFunc{cos} & Cosine\\
\hlnkFunc{cosh} & Hyperbolic Cosine\\
\hlnkFunc{exp} & Exponential\\
\hlnkFunc{exp10} & Exponential Base 10\\
\hlnkFunc{log} & Natural Log\\
\hlnkFunc{log10} & Base 10 Log\\
\hlnkFunc{sin} & Sine \\
\hlnkFunc{sinh} & Hyperbolic Sine\\
\hlnkFunc{sqrt} & Square Root\\
\hlnkFunc{tan} & Tangent\\
\hlnkFunc{tanh} & Hyperbolic Tangent\\
\end{tabular}
\end{center}
\label{default}
\end{table}%

    \subsubsection*{Unary Operations\hspace*{\fill}\hyperlink{ElementwiseOperations}{(up)}\hypertarget{unaryOperations}{}}\addcontentsline{toc}{subsubsection}{Unary Operations}
Unary operations involve calculations on a single \ttbf{view}. Functions which involve a calculation where the answer is a scalar, such as \ttbf{sumval} generally have a \ttbf{val} as part of the root name. 
\begin{table}[H]
\caption{Unary Operations}
\label{tab:unaryOperations}
\begin{center}
\begin{tabular}{|l|l|}\hline
\hlnkFunc{arg} & Argument\\
\hlnkFunc{ceil} & Ceiling\\
\hlnkFunc{conj} & Conjugate\\
\hlnkFunc{cumsum} & Cumulative Sum\\
\hlnkFunc{euler} & Euler\\
\hlnkFunc{floor} & Floor\\
\hlnkFunc{mag} & Magnitude\\
\hlnkFunc{cmagsq} & Complex Magnitude Squared\\
\hlnkFunc{meanval} & Mean Value\\
\hlnkFunc{meansqval} & Mean Square Value\\
\hlnkFunc{modulate} & Modulate\\
\hlnkFunc{neg} & Negate\\
\hlnkFunc{recip} & Reciprocal\\
\hlnkFunc{round} & Round\\
\hlnkFunc{rsqrt} & reciprocal Square Root\\
\hlnkFunc{sq} & Square\\
\hlnkFunc{sumval} & Sum Value\\
\hlnkFunc{sumsqval} & Sum of Squares Value\\
\hline\end{tabular}
\end{center}
%\label{default}
\end{table}%

\subsection*{Signal Processing Functions}
\subsection*{Linear Algebra Functions}
\subsection*{Implementation Dependent Input and Output}
   \ttbf{JVSIP} does not support implementation dependent IO at this time.
\subsection*{VSIPL Interpolation}

\subsection*{VSIPL Permute}
\subsection*{VSIPL Sort}


\section*{VSIP Types}
This section covers the enumerated types and special structures. These are declared in the public header file \ilCode{vsip.h}. 
\section*{JVSIP Function List}\addcontentsline{roc}{section}{JVSIP Function List} 



\afunc{acos}{Inverse Cosine. An elementary math function. See table \ref{tab:elementaryMath}.}
\cvsiplh
\newline \hspace*{.8cm} \vspace*{.1cm} \textbf{Available Functions }
\newline \hspace*{1.1cm} {
\ttfamily
\begin{tabular}[H]{l}
vsip\_scalar\_f vsip\_acos\_f(vsip\_scalar\_f a);\\
vsip\_scalar\_d vsip\_acos\_d(vsip\_scalar\_d a);\\
void vsip\_macos\_d(\\*
\hspace{1cm}const vsip\_mview\_d*, const vsip\_mview\_d*);\\
void vsip\_macos\_f(\\*
\hspace{1cm}const vsip\_mview\_f*, const vsip\_mview\_f*);\\
void vsip\_vacos\_d(\\*
\hspace{1cm}const vsip\_vview\_d*, const vsip\_vview\_d*);\\
void vsip\_vacos\_f(\\*
\hspace{1cm}const vsip\_vview\_f*, const vsip\_vview\_f*);\\
\end{tabular}
}
\pyjvsiph
\viewmthd{yes}{yes}{yes}{inOut.acos}
\apyfunc{yes}{out = acos(in,out)}
\pyComment{\item{The \ttbf{acos} function works much the same as the C VSIPL version except that a convenience pointer to the output view is returned. }
\item{This may be done in-place if \ttbf{in==out}.}}\afuncT{arg}{Compute the radian value argument of complex elements in the interval $[-\pi,\pi]$. An Unary Operation.}{unaryOperations}
\\\cvsiplh
\begin{cfuncs}
vsip\_scalar\_d vsip\_arg\_d(vsip\_cscalar\_d);\Bs\\
vsip\_scalar\_f vsip\_arg\_f(vsip\_cscalar\_f);\Bs\\
void vsip\_marg\_d(const~vsip\_cmview\_d*, const~vsip\_mview\_d*);\Bs\\
void vsip\_marg\_f(const~vsip\_cmview\_f*, const~vsip\_mview\_f*);\Bs\\
void vsip\_varg\_d(const~vsip\_cvview\_d*, const~vsip\_vview\_d*);\Bs\\
void vsip\_varg\_f(const~vsip\_cvview\_f*, const~vsip\_vview\_f*);\Bs\\
\end{cfuncs}
\pyjvsiph
\viewmthd{yes}{yes}{No}{out=in.arg}
\apyfunc{yes}{out = arg(in,out)}
\\ \hspace*{.8cm}\textbf{Comment}\\
\hspace*{.8cm}\parbox{11cm}{\vspace*{.2cm}
\begin{itemize}
\item{Since \ttbf{arg} takes a view of \emph{depth} complex and outputs to a view of \emph{depth} real of the same \emph{shape} and \emph{precision} as the input view the \ttbf{arg} method will create a view of the proper type and size and return it.}
\item{The \ttbf{arg} function works the same as the C VSIPL function except a convenience pointer is returned to the output view}
\item{For the function limited in-place functionality exists with replacement of the real or imaginary view of the input with the output. For instance \ilCode{out=arg(in,in.realview)} works fine.}
\end{itemize}}
\afuncT{asin}{Inverse Cosine. An elementary math function.}{elementaryMath}
\\\cvsiplh
\afh
\\\hspace*{.04\textwidth} {
\ttfamily
\begin{tabular}[H]{l}
vsip\_scalar\_f vsip\_asin\_f(vsip\_scalar\_f a);\\
vsip\_scalar\_d vsip\_asin\_d(vsip\_scalar\_d a);\\
void vsip\_masin\_d(const vsip\_mview\_d*, const vsip\_mview\_d*);\\
void vsip\_masin\_f(const vsip\_mview\_f*, const vsip\_mview\_f*);\\
void vsip\_vasin\_d(const vsip\_vview\_d*, const vsip\_vview\_d*);\\
void vsip\_vasin\_f(const vsip\_vview\_f*, const vsip\_vview\_f*);\\
\end{tabular}
}
\\\pyjvsiph
\viewmthd{yes}{yes}{yes}{inOut.asin}
\apyfunc{yes}{out = asin(in,out)}
\pyComment{\item{The \ttbf{asin} function works much the same as the C VSIPL version except that a convenience pointer to the output view is returned. This may be done in-place if \ttbf{in==out}.}}
\afuncT{atan}{Computes the principal radian value, $[-\pi/2,\pi/2]$, of the arctangent for each element of a \ttbf{view}.}{elementaryMath}
\\\cvsiplh 
\afh 
\\\hspace*{.04\textwidth} {
\ttfamily
\begin{tabular}[H]{l}
vsip\_scalar\_f vsip\_atan\_f(vsip\_scalar\_f);\\
vsip\_scalar\_d vsip\_atan\_d(vsip\_scalar\_d);\\
void vsip\_matan\_d(const vsip\_mview\_d*, const vsip\_mview\_d*);\\
void vsip\_matan\_f(const vsip\_mview\_f*, const vsip\_mview\_f*);\\
void vsip\_vatan\_d(const vsip\_vview\_d*, const vsip\_vview\_d*);\\
void vsip\_vatan\_f(const vsip\_vview\_f*, const vsip\_vview\_f*);\\
\end{tabular}
}
\\\pyjvsiph
\viewmthd{yes}{yes}{yes}{inOut.atan}
\apyfunc{yes}{out = atan(in,out)}
\pyComment{\item{The \ttbf{atan} function works much the same as the C VSIPL version except that a convenience pointer to the output view is returned. This may be done in-place if \ttbf{in==out}.}}\afuncT{atan2}{Arctangent of Two Arguments; An elementwise function. Computes the four quadrant radian value, $[-\pi,\pi]$, of the arctangent of the ratio of the corresponding elements of two input views.}{elementaryMath}
\\\cvsiplh
\hspace*{.8cm} \vspace*{.1cm} \textbf{Available Functions } \\
\hspace*{1cm}
\texttt{
\begin{tabular}[H]{l}
vsip\_scalar\_d vsip\_atan2\_d(vsip\_scalar\_d, vsip\_scalar\_d);\\
vsip\_scalar\_f vsip\_atan2\_f(vsip\_scalar\_f, vsip\_scalar\_f);\\
void vsip\_matan2\_d(const vsip\_mview\_d*, const vsip\_mview\_d*, const vsip\_mview\_d*);\\
void vsip\_matan2\_f(const vsip\_mview\_f*, const vsip\_mview\_f*, const vsip\_mview\_f*);\\
void vsip\_vatan2\_d(const vsip\_vview\_d*, const vsip\_vview\_d*, const vsip\_vview\_d*);\\
void vsip\_vatan2\_f(const vsip\_vview\_f*, const vsip\_vview\_f*, const vsip\_vview\_f*);\\
\end{tabular}
}
\\\pyjvsiph
\afuncT{ceil}{Ceiling. An unary operation.}{unaryOperations}
\\\cvsiplh
\\ \hspace*{.8cm} \vspace*{.1cm} \textbf{Available Functions }
\\ \hspace*{1.1cm} {
\ttfamily
\begin{tabular}[H]{l}
void vsip\_mceil\_d\_d(const vsip\_mview\_d*, const vsip\_mview\_d*);\\
void vsip\_mceil\_d\_i(const vsip\_mview\_d*, const vsip\_mview\_i*);\\
void vsip\_mceil\_f\_f(const vsip\_mview\_f*, const vsip\_mview\_f*);\\
void vsip\_mceil\_f\_i(const vsip\_mview\_f*, const vsip\_mview\_i*);\\
void vsip\_vceil\_d\_d(const vsip\_vview\_d*, const vsip\_vview\_d*);\\
void vsip\_vceil\_d\_i(const vsip\_vview\_d*, const vsip\_vview\_i*);\\
void vsip\_vceil\_f\_f(const vsip\_vview\_f*, const vsip\_vview\_f*);\\
void vsip\_vceil\_f\_i(const vsip\_vview\_f*, const vsip\_vview\_i*);\\
\end{tabular}
}
\\\pyjvsiph
\pyjvComment{
\item{Ceiling }
}
\afuncT{cos}{Cosine; An elementary math function.}{elementaryMath}
\\\cvsiplh
\afh
\\\hspace*{.04\textwidth} {
\ttfamily
\begin{tabular}[H]{l}
vsip\_scalar\_f vsip\_cos\_f(vsip\_scalar\_f a);\\
vsip\_scalar\_d vsip\_cos\_d(vsip\_scalar\_d a);\\
void vsip\_mcos\_d(const vsip\_mview\_d*, const vsip\_mview\_d*);\\
void vsip\_mcos\_f(const vsip\_mview\_f*, const vsip\_mview\_f*);\\
void vsip\_vcos\_d(const vsip\_vview\_d*, const vsip\_vview\_d*);\\
void vsip\_vcos\_f(const vsip\_vview\_f*, const vsip\_vview\_f*);\\
\end{tabular}
}
\\\pyjvsiph
\viewmthd{yes}{yes}{yes}{inOut.cos}
\apyfunc{yes}{out = cos(in,out)}
\pyComment{
\item{The \ttbf{cos} function works much the same as the C VSIPL version except that a convenience pointer to the output view is returned. This may be done in-place if \ttbf{in==out}.}}
\afuncT{cosh}{Hyperbolic Cosine; An elementwise function}{elementaryMath}
\\\cvsiplh
\afh
\\\hspace*{.04\textwidth} {
\ttfamily
\begin{tabular}[H]{l}
vsip\_scalar\_f vsip\_cosh\_f(vsip\_scalar\_f a);\\
vsip\_scalar\_d vsip\_cosh\_d(vsip\_scalar\_d a);\\
void vsip\_mcosh\_d(const vsip\_mview\_d*, const vsip\_mview\_d*);\\
void vsip\_mcosh\_f(const vsip\_mview\_f*, const vsip\_mview\_f*);\\
void vsip\_vcosh\_d(const vsip\_vview\_d*, const vsip\_vview\_d*);\\
void vsip\_vcosh\_f(const vsip\_vview\_f*, const vsip\_vview\_f*);\\
\end{tabular}
}
\\\pyjvsiph
\viewmthd{yes}{yes}{yes}{inOut.cosh}
\apyfunc{yes}{out = cosh(in,out)}
\pyComment{
\item{The \ttbf{cosh} function works much the same as the C VSIPL version except that a convenience pointer to the output view is returned. This may be done in-place if \ttbf{in==out}.}}
\afuncT{ceil}{Ceiling. An unary operation.}{unaryOperations}
\\\cvsiplh
\\ \hspace*{.8cm} \vspace*{.1cm} \textbf{Available Functions }
\\ \hspace*{1.1cm} {
\ttfamily
\begin{tabular}[H]{l}
void vsip\_mceil\_d\_d(const vsip\_mview\_d*, const vsip\_mview\_d*);\\
void vsip\_mceil\_d\_i(const vsip\_mview\_d*, const vsip\_mview\_i*);\\
void vsip\_mceil\_f\_f(const vsip\_mview\_f*, const vsip\_mview\_f*);\\
void vsip\_mceil\_f\_i(const vsip\_mview\_f*, const vsip\_mview\_i*);\\
void vsip\_vceil\_d\_d(const vsip\_vview\_d*, const vsip\_vview\_d*);\\
void vsip\_vceil\_d\_i(const vsip\_vview\_d*, const vsip\_vview\_i*);\\
void vsip\_vceil\_f\_f(const vsip\_vview\_f*, const vsip\_vview\_f*);\\
void vsip\_vceil\_f\_i(const vsip\_vview\_f*, const vsip\_vview\_i*);\\
\end{tabular}
}
\\\pyjvsiph
\pyjvComment{
\item{Ceiling }
}\afuncT{conj}{Conjugate each element in a view. An Unary Operations.}{unaryOperations}
\\\cvsiplh
\afh
\\\hspace*{.04\textwidth} {
\ttfamily
\begin{tabular}[H]{l}
vsip\_cscalar\_d vsip\_conj\_d(vsip\_cscalar\_d);\\
vsip\_cscalar\_f vsip\_conj\_f(vsip\_cscalar\_f);\\
void vsip\_cmconj\_d(\\*\hspace*{1cm}const vsip\_cmview\_d*, const vsip\_cmview\_d*);\\
void vsip\_cmconj\_f(\\*\hspace*{1cm}const vsip\_cmview\_f*, const vsip\_cmview\_f*);\\
void vsip\_cvconj\_d(\\*\hspace*{1cm}const vsip\_cvview\_d*, const vsip\_cvview\_d*);\\
void vsip\_cvconj\_f(\\*\hspace*{1cm}const vsip\_cvview\_f*, const vsip\_cvview\_f*);\\
\end{tabular}
}
\\\pyjvsiph
\viewmthd{yes}{yes}{yes}{inOut.conj}
\apyfunc{yes}{out = conj(in,out)}
\pyComment{

\item{The \ttbf{conj} function works much the same as the C VSIPL version except that a convenience pointer to the output view is returned. This may be done in-place if \ttbf{in==out}.}
\item{If the calling \ttbf{view} for the \ttbf{conj} method is real no error is generated. This case is basically a no operation. This is not true for the \ttbf{conj} function call which will generate an assert error as an unsupported type.}
}\afunc{cumsum}{Cumulative Sum}
\index{Cumulative Sum}
\\\cvsiplh
\\\pyjvsiph\afunc{copy}{Copy Data between two views. Some mixed types are supported so this method can be used to produce a copy of data of a new precision}
\\\cvsiplh
\newline \hspace*{.8cm} \vspace*{.1cm} \textbf{Available Functions }
\newline \hspace*{1cm} {\ttfamily
\begin{tabular}[H]{l}
void vsip\_cmcopy\_d\_d(\\*
\hspace{1cm}const vsip\_cmview\_d*, const vsip\_cmview\_d*);\\
void vsip\_cmcopy\_d\_f(\\*
\hspace{1cm}const vsip\_cmview\_d*, const vsip\_cmview\_f*);\\
void vsip\_cmcopy\_f\_d(\\*
\hspace{1cm}const vsip\_cmview\_f*, const vsip\_cmview\_d*);\\
void vsip\_cmcopy\_f\_f(\\*
\hspace{1cm}const vsip\_cmview\_f*, const vsip\_cmview\_f*);\\
void vsip\_cvcopy\_d\_d\\*
\hspace{1cm}(const vsip\_cvview\_d*, const vsip\_cvview\_d*);\\
$\cdots$  \emph{etc.} \end{tabular}
}
\newline \hspace*{1cm}
\parbox{11cm}{There are many copy functions. To see all supported search the \ilCode{vsip.h} header file.\footnotemark}
\footnotetext{For instance \ttbf{grep copy\_ vsip.h} will list all available copy functions.}
\\\pyjvsiph
\viewmthd{yes}{yes}{no}{\parbox[t]{4cm}{out=in.copy\\out=in.copyrm\\out=in.copycm}}
\newline\hspace*{1cm}\parbox{11cm}{The \ttbf{copy} method creates a new view and data space that is the same shape, precision and depth as the input view and copies the data from the \ilCode{in} view to the \ilCode{out} view. The block in the \ilCode{out} view will be the exact size needed to hold the data and will be unit stride along the major direction of the \ilCode{in} view.\\The {\texttt{\bfseries{copycm}}} method is the same as the \ilCode{copy} method except the output view will always be row major independent of the input views major direction.\\The \ttbf{copyrm} method is the same as the \ilCode{copy} method except the output view will always be column major independent of the input views major direction.\\If the input view is a vector the three copy methods have identical results.}
\newline
\apyfunc{yes}{out = copy(in,out)}
\newline\hspace*{1cm}\parbox{11cm}{The \ttbf{copy} function works much the same as the C VSIPL version except that a convenience pointer to the output view is returned.}

\afunc{euler}{Euler}
\\\cvsiplh
\\\pyjvsiph\afuncT{exp}{Exponential; An elementwise function}{elementaryMath}
\\\cvsiplh
\afh
\\\hspace*{.04\textwidth} {
\ttfamily
\begin{tabular}[H]{l}
vsip\_cscalar\_d vsip\_cexp\_d(vsip\_cscalar\_d);\\
vsip\_cscalar\_f vsip\_cexp\_f(vsip\_cscalar\_f);\\
void vsip\_cmexp\_d(const vsip\_cmview\_d*, const vsip\_cmview\_d*);\\
void vsip\_cmexp\_f(const vsip\_cmview\_f*, const vsip\_cmview\_f*);\\
void vsip\_cvexp\_d(const vsip\_cvview\_d*, const vsip\_cvview\_d*);\\
void vsip\_cvexp\_f(const vsip\_cvview\_f*, const vsip\_cvview\_f*);\\
void vsip\_mexp\_d(const vsip\_mview\_d*, const vsip\_mview\_d*);\\
void vsip\_mexp\_f(const vsip\_mview\_f*, const vsip\_mview\_f*);\\
void vsip\_vexp\_d(const vsip\_vview\_d*, const vsip\_vview\_d*);\\
void vsip\_vexp\_f(const vsip\_vview\_f*, const vsip\_vview\_f*);\\
\end{tabular}
}
\\\pyjvsiph
\viewmthd{yes}{yes}{yes}{inOut.exp}
\apyfunc{yes}{out = exp(in,out)}
\pyComment{\item{The \ttbf{exp} function works much the same as the C VSIPL version except that a convenience pointer to the output view is returned. This may be done in-place if \ttbf{in==out}.}}
\afuncT{exp10}{exp10onential Base 10; An elementwise function}{elementaryMath}
\\\cvsiplh
\\ \hspace*{.8cm} \vspace*{.1cm} \textbf{Available Functions }
\\ \hspace*{1.1cm} {
\ttfamily
\begin{tabular}[H]{l}
vsip\_scalar\_d vsip\_exp10\_d(vsip\_scalar\_d);\\
vsip\_scalar\_f vsip\_exp10\_f(vsip\_scalar\_f);\\
void vsip\_mexp10\_d(const vsip\_mview\_d*, const vsip\_mview\_d*);\\
void vsip\_mexp10\_f(const vsip\_mview\_f*, const vsip\_mview\_f*);\\
void vsip\_vexp10\_d(const vsip\_vview\_d*, const vsip\_vview\_d*);\\
void vsip\_vexp10\_f(const vsip\_vview\_f*, const vsip\_vview\_f*);\\
\end{tabular}
}
\\\pyjvsiph
\viewmthd{yes}{yes}{yes}{inOut.exp10}
\apyfunc{yes}{out = exp10(in,out)}
\pyComment{\item{The \ttbf{exp10} function works much the same as the C VSIPL version except that a convenience pointer to the output view is returned. This may be done in-place if \ttbf{in==out}.}}

\afunc{floor}{For each element in the input \ttbf{view} round to the largest integral value not greater than the input. An unary operation. See table \ref{tab:unaryOperations}.}
\\\cvsiplh
\\\pyjvsiph
\pyjvComment{
\item{The \ilCode{floor} function is not supported in \jv at this time}
}
\afuncT{mag}{Arctangent of Two Arguments; An elementwise function}{unaryOperations}
\\\cvsiplh
\\\pyjvsiph\afuncT{magsq}{Arctangent of Two Arguments; An elementwise function}{unaryOperations}
\\\cvsiplh
\afh
\\\hspace*{.04\textwidth} {
\ttfamily
}
\\\pyjvsiph\afuncT{meanval}{Returns the mean value of all the elements of a view.}{unaryOperations}
\\\cvsiplh
\afh
\\\hspace*{.04\textwidth} {
\ttfamily
\begin{tabular}[H]{l}
vsip\_cscalar\_d vsip\_cmmeanval\_d(const vsip\_cmview\_d*);\\
vsip\_cscalar\_d vsip\_cvmeanval\_d(const vsip\_cvview\_d*);\\
vsip\_cscalar\_f vsip\_cmmeanval\_f(const vsip\_cmview\_f*);\\
vsip\_cscalar\_f vsip\_cvmeanval\_f(const vsip\_cvview\_f*);\\
vsip\_scalar\_d vsip\_mmeanval\_d(const vsip\_mview\_d*);\\
vsip\_scalar\_d vsip\_vmeanval\_d(const vsip\_vview\_d*);\\
vsip\_scalar\_f vsip\_mmeanval\_f(const vsip\_mview\_f*);\\
vsip\_scalar\_f vsip\_vmeanval\_f(const vsip\_vview\_f*);\\
\end{tabular}
}
\\\pyjvsiph
\viewmthd{Yes}{Yes}{NA}{m=in.meanval}
\apyfunc{No}{NA}
\pyComment{\item{There seemed to be no reason to include this as a separate function for \pyjv}}\afuncT{meansqval}{Returns the mean value of all the elements of a view.}{unaryOperations}
\\\cvsiplh
\newline \hspace*{.8cm} \vspace*{.1cm} \textbf{Available Functions }
\newline \hspace*{1.1cm} {
\ttfamily
\begin{tabular}[H]{l}
vsip\_scalar\_d vsip\_cmmeansqval\_d(const vsip\_cmview\_d*);\\
vsip\_scalar\_d vsip\_cvmeansqval\_d(const vsip\_cvview\_d*);\\
vsip\_scalar\_d vsip\_mmeansqval\_d(const vsip\_mview\_d*);\\
vsip\_scalar\_d vsip\_vmeansqval\_d(const vsip\_vview\_d*);\\
vsip\_scalar\_f vsip\_cmmeansqval\_f(const vsip\_cmview\_f*);\\
vsip\_scalar\_f vsip\_cvmeansqval\_f(const vsip\_cvview\_f*);\\
vsip\_scalar\_f vsip\_mmeansqval\_f(const vsip\_mview\_f*);\\
vsip\_scalar\_f vsip\_vmeansqval\_f(const vsip\_vview\_f*);\\
\end{tabular}
}
\\\pyjvsiph
\viewmthd{Yes}{Yes}{NA}{msq=in.meansqval}
\apyfunc{No}{NA}
\pyComment{\item{There seemed to be no reason to include this as a separate function for \pyjv}}\afuncT{modulate}{Computes the modulation of a real vector by a specified complex frequency.}{unaryOperations}
\\\cvsiplh
\afh
\\\hspace*{.04\textwidth} {
\ttfamily
\begin{tabular}[H]{l}
vsip\_scalar\_d vsip\_cvmodulate\_d(\\*\hspace{.6cm}const vsip\_cvview\_d*, vsip\_scalar\_d, \\*\hspace{.6cm}vsip\_scalar\_d, const vsip\_cvview\_d*);\\
vsip\_scalar\_d vsip\_vmodulate\_d(\\*\hspace{.6cm}const vsip\_vview\_d*, vsip\_scalar\_d, \\*\hspace{.6cm}vsip\_scalar\_d, const vsip\_cvview\_d*);\\
vsip\_scalar\_f vsip\_cvmodulate\_f(\\*\hspace{.6cm}const vsip\_cvview\_f*, vsip\_scalar\_f, \\*\hspace{.6cm}vsip\_scalar\_f, const vsip\_cvview\_f*);\\
vsip\_scalar\_f vsip\_vmodulate\_f(\\*\hspace{.6cm}const vsip\_vview\_f*, vsip\_scalar\_f, \\*\hspace{.6cm}vsip\_scalar\_f, const vsip\_cvview\_f*);\\
\end{tabular}
}
\\\pyjvsiph
\viewmthd{No}{NA}{NA}{NA}
\apyfunc{Yes}{phiNew,out=modulate(in,nu,phi,out)}
\pyComment{{\item{Note \ttbf{phi} is the initial phase and the final phase is returned as \ttbf{phiNew}. For \pyjv we also return a convenience copy of the output vector}}}
\afunc{neg}{Computes the reciprocal for each element of a \ttbf{view}. An elementwise function. See unary operations table \ref{tab:unaryOperations}}
\cvsiplh
\newline \hspace*{.8cm} \vspace*{.1cm} \textbf{Available Functions }
\newline \hspace*{1.1cm} {
\ttfamily
\begin{tabular}[H]{l}
vsip\_cscalar\_d vsip\_cneg\_d(vsip\_cscalar\_d);\\
vsip\_cscalar\_f vsip\_cneg\_f(vsip\_cscalar\_f);\\
void vsip\_cmneg\_d(\\*\hspace{.6cm}const vsip\_cmview\_d*, const vsip\_cmview\_d*);\\
void vsip\_cmneg\_f(\\*\hspace{.6cm}const vsip\_cmview\_f*, const vsip\_cmview\_f*);\\
void vsip\_cvneg\_d(\\*\hspace{.6cm}const vsip\_cvview\_d*, const vsip\_cvview\_d*);\\
void vsip\_cvneg\_f(\\*\hspace{.6cm}const vsip\_cvview\_f*, const vsip\_cvview\_f*);\\
void vsip\_mneg\_d(\\*\hspace{.6cm}const vsip\_mview\_d*, const vsip\_mview\_d*);\\
void vsip\_mneg\_f(\\*\hspace{.6cm}const vsip\_mview\_f*, const vsip\_mview\_f*);\\
void vsip\_vneg\_d(\\*\hspace{.6cm}const vsip\_vview\_d*, const vsip\_vview\_d*);\\
void vsip\_vneg\_f(\\*\hspace{.6cm}const vsip\_vview\_f*, const vsip\_vview\_f*);\\
void vsip\_vneg\_i(\\*\hspace{.6cm}const vsip\_vview\_i*, const vsip\_vview\_i*);\\
void vsip\_vneg\_si(\\*\hspace{.6cm}const vsip\_vview\_si*, const vsip\_vview\_si*);\\
\end{tabular}
}
\pyjvsiph
\afunc{recip}{Computes the reciprocal for each element of a \ttbf{view}. An elementwise function. See unary operations table \ref{tab:unaryOperations}}
\cvsiplh
\newline \hspace*{.8cm} \vspace*{.1cm} \textbf{Available Functions }
\newline \hspace*{1.1cm} {
\ttfamily
\begin{tabular}[H]{l}
vsip\_cscalar\_d vsip\_crecip\_d(vsip\_cscalar\_d);\\
vsip\_cscalar\_f vsip\_crecip\_f(vsip\_cscalar\_f);\\
void vsip\_cmrecip\_d(\\*\hspace{.6cm}const vsip\_cmview\_d*, const vsip\_cmview\_d*);\\
void vsip\_cmrecip\_f(\\*\hspace{.6cm}const vsip\_cmview\_f*, const vsip\_cmview\_f*);\\
void vsip\_cvrecip\_d(\\*\hspace{.6cm}const vsip\_cvview\_d*, const vsip\_cvview\_d*);\\
void vsip\_cvrecip\_f(\\*\hspace{.6cm}const vsip\_cvview\_f*, const vsip\_cvview\_f*);\\
void vsip\_mrecip\_d(\\*\hspace{.6cm}const vsip\_mview\_d*, const vsip\_mview\_d*);\\
void vsip\_mrecip\_f(\\*\hspace{.6cm}const vsip\_mview\_f*, const vsip\_mview\_f*);\\
void vsip\_vrecip\_d(\\*\hspace{.6cm}const vsip\_vview\_d*, const vsip\_vview\_d*);\\
void vsip\_vrecip\_f(\\*\hspace{.6cm}const vsip\_vview\_f*, const vsip\_vview\_f*);\\
\end{tabular}
}
\pyjvsiph\afuncT{round}{Round to nearest integral value; An elementwise function. }{unaryOperations}
\\\cvsiplh
\afh
\\\hspace*{.04\textwidth} {
\ttfamily
}
\\\pyjvsiph
\pyjvComment{
\item{The \ilCode{round} function is not supported in \jv{} at this time}
}
\afuncT{rsqrt}{Reciprocal square root; An elementwise function.}{unaryOperations}
\\\cvsiplh
\afh
\\\hspace*{.04\textwidth} {
\ttfamily
}
\\\pyjvsiph
\afuncT{sin}{Sine; An element-wise function. Input \ttbf{view} elements are assumed to be in radians.}{elementaryMath}
\\\cvsiplh
\newline \hspace*{.8cm} \vspace*{.1cm} \textbf{Available Functions }
\newline \hspace*{1.1cm} {
\ttfamily
\begin{tabular}[H]{l}
vsip\_scalar\_f vsip\_sin\_f(vsip\_scalar\_f a);\\
vsip\_scalar\_d vsip\_sin\_d(vsip\_scalar\_d a);\\
void vsip\_msin\_d(\\*
\hspace{1cm}const vsip\_mview\_d*, const vsip\_mview\_d*);\\
void vsip\_msin\_f(\\*
\hspace{1cm}const vsip\_mview\_f*, const vsip\_mview\_f*);\\
void vsip\_vsin\_d(\\*
\hspace{1cm}const vsip\_vview\_d*, const vsip\_vview\_d*);\\
void vsip\_vsin\_f(\\*
\hspace{1cm}const vsip\_vview\_f*, const vsip\_vview\_f*);\\
\end{tabular}
}
\\\pyjvsiph
\viewmthd{yes}{yes}{yes}{inOut.sin}
\apyfunc{yes}{out = sin(in,out)}
\newline\hspace*{1.2cm}\parbox{10.8cm}{\vspace*{.1cm}The \ttbf{sin} function works much the same as the C VSIPL version except that a convenience pointer to the output view is returned. This may be done in-place if \ttbf{in==out}.}
\afunc{sinh}{Hyperbolic Sine; An elementwise function. See elementary math functions table \ref{tab:elementaryMath}.}
\cvsiplh
\newline \hspace*{.8cm} \vspace*{.1cm} \textbf{Available Functions }
\newline \hspace*{1.1cm} {
\ttfamily
\begin{tabular}[H]{l}
vsip\_scalar\_f vsip\_sinh\_f(vsip\_scalar\_f a);\\
vsip\_scalar\_d vsip\_sinh\_d(vsip\_scalar\_d a);\\
void vsip\_msinh\_d(\\*
\hspace{1cm}const vsip\_mview\_d*, const vsip\_mview\_d*);\\
void vsip\_msinh\_f(\\*
\hspace{1cm}const vsip\_mview\_f*, const vsip\_mview\_f*);\\
void vsip\_vsinh\_d(\\*
\hspace{1cm}const vsip\_vview\_d*, const vsip\_vview\_d*);\\
void vsip\_vsinh\_f(\\*
\hspace{1cm}const vsip\_vview\_f*, const vsip\_vview\_f*);\\
\end{tabular}
}
\pyjvsiph
\viewmthd{yes}{yes}{yes}{inOut.sinh}
\apyfunc{yes}{out = sinh(in,out)}
\newline\hspace*{1.2cm}\parbox{10.8cm}{\vspace*{.1cm}The \ttbf{sinh} function works much the same as the C VSIPL version except that a convenience pointer to the output view is returned. This may be done in-place if \ttbf{in==out}.}
\afuncT{sqrt}{Square Root; An elementwise function.}{elementaryMath}
\\\cvsiplh
\\\pyjvsiph
\afuncT{sq}{Square each element in a \ttbf{view}.} {unaryOperations}
\\\cvsiplh
\afh
\\\hspace*{.04\textwidth} {
\ttfamily
}
\\\pyjvsiph\afuncT{sumval}{Returns the sum of the the elements of a \ttbf{view}. Does not modify input.}{unaryOperations}
\\\cvsiplh
\newline \hspace*{.8cm} \vspace*{.1cm} \textbf{Available Functions }
\newline \hspace*{1.1cm} {
\ttfamily
\begin{tabular}[H]{l}
vsip\_cscalar\_d vsip\_cmsumval\_d(const vsip\_cmview\_d*);\\
vsip\_cscalar\_d vsip\_cvsumval\_d(const vsip\_cvview\_d*);\\
vsip\_cscalar\_f vsip\_cmsumval\_f(const vsip\_cmview\_f*);\\
vsip\_cscalar\_f vsip\_cvsumval\_f(const vsip\_cvview\_f*);\\
vsip\_scalar\_d vsip\_msumval\_d(const vsip\_mview\_d*);\\
vsip\_scalar\_d vsip\_vsumval\_d(const vsip\_vview\_d*);\\
vsip\_scalar\_f vsip\_msumval\_f(const vsip\_mview\_f*);\\
vsip\_scalar\_f vsip\_vsumval\_f(const vsip\_vview\_f*);\\
vsip\_scalar\_i vsip\_vsumval\_i(const vsip\_vview\_i*);\\
vsip\_scalar\_si vsip\_vsumval\_si(const vsip\_vview\_si*);\\
vsip\_scalar\_uc vsip\_vsumval\_uc(const vsip\_vview\_uc*);\\
vsip\_scalar\_vi vsip\_msumval\_bl(const vsip\_mview\_bl*);\\
vsip\_scalar\_vi vsip\_vsumval\_bl(const vsip\_vview\_bl*);\\
\end{tabular}
}
\\\pyjvsiph\afunc{sumsqval}{Returns the sum of the squares of all the elements of a \ttbf{view}. Does not modify input. See table \ref{tab:unaryOperations}}
\cvsiplh
\newline \hspace*{.8cm} \vspace*{.1cm} \textbf{Available Functions }
\newline \hspace*{1.1cm} {
\ttfamily
\begin{tabular}[H]{l}
vsip\_scalar\_d vsip\_msumsqval\_d(const vsip\_mview\_d* );\\
vsip\_scalar\_d vsip\_vsumsqval\_d(const vsip\_vview\_d* );\\
vsip\_scalar\_f vsip\_msumsqval\_f(const vsip\_mview\_f* );\\
vsip\_scalar\_f vsip\_vsumsqval\_f(const vsip\_vview\_f* );\\
\end{tabular}
}
\pyjvsiph
\viewmthd{yes}{yes}{NA}{aValue=in.sumsqval}
\apyfunc{No}{}
\newline \hspace*{.8cm} \textbf{Comment}
\newline\hspace*{.9cm}\parbox{10.8cm}{\vspace*{.1cm}Since the \ttbf{sumsqval} function returns a scalar without modifying the \ttbf{view} there seemed little point in supporting this as a separate function call for \pyjv.}

\afuncT{tan}{Tangent; An elementwise function. Input \ttbf{view} elements are assumed to be in radians.}{elementaryMath}
\\\cvsiplh
\afh
\\\hspace*{.04\textwidth} {
\ttfamily
\begin{tabular}[H]{l}
vsip\_scalar\_f vsip\_tan\_f(vsip\_scalar\_f a);\\
vsip\_scalar\_d vsip\_tan\_d(vsip\_scalar\_d a);\\
void vsip\_mtan\_d(const vsip\_mview\_d*, const vsip\_mview\_d*);\\
void vsip\_mtan\_f(const vsip\_mview\_f*, const vsip\_mview\_f*);\\
void vsip\_vtan\_d(const vsip\_vview\_d*, const vsip\_vview\_d*);\\
void vsip\_vtan\_f(const vsip\_vview\_f*, const vsip\_vview\_f*);\\
\end{tabular}
}
\\\pyjvsiph
\viewmthd{yes}{yes}{yes}{inOut.tan}
\apyfunc{yes}{out = tan(in,out)}
\pyComment{
\item{The \ttbf{tan} function works much the same as the C VSIPL version except that a convenience pointer to the output view is returned. This may be done in-place if \ttbf{in==out}.}}
\afunc{tanh}{Hyperbolic Tangent; An elementwise function. See elementary math functions table \ref{tab:elementaryMath}.}
\cvsiplh
\newline \hspace*{.8cm} \vspace*{.1cm} \textbf{Available Functions }
\newline \hspace*{1.1cm} {
\ttfamily
\begin{tabular}[H]{l}
vsip\_scalar\_f vsip\_tanh\_f(vsip\_scalar\_f);\\
vsip\_scalar\_d vsip\_tanh\_d(vsip\_scalar\_d);\\
void vsip\_mtanh\_d(\\*\hspace{1cm}const vsip\_mview\_d*, const vsip\_mview\_d*);\\
void vsip\_mtanh\_f(\\*\hspace{1cm}const vsip\_mview\_f*, const vsip\_mview\_f*);\\
void vsip\_vtanh\_d(\\*\hspace{1cm}const vsip\_vview\_d*, const vsip\_vview\_d*);\\
void vsip\_vtanh\_f(\\*\hspace{1cm}const vsip\_vview\_f*, const vsip\_vview\_f*);\\
\end{tabular}
}
\pyjvsiph

