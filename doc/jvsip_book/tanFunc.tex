\afuncT{tan}{Tangent; An elementwise function. Input \ttbf{view} elements are assumed to be in radians.}{elementaryMath}
\\\cvsiplh
\begin{cfuncs}
vsip\_scalar\_f~vsip\_tan\_f(vsip\_scalar\_f~a);\\
vsip\_scalar\_d~vsip\_tan\_d(vsip\_scalar\_d~a);\\
void~vsip\_mtan\_d(const~vsip\_mview\_d*, const~vsip\_mview\_d*);\\
void~vsip\_mtan\_f(const~vsip\_mview\_f*, const~vsip\_mview\_f*);\\
void~vsip\_vtan\_d(const~vsip\_vview\_d*, const~vsip\_vview\_d*);\\
void~vsip\_vtan\_f(const~vsip\_vview\_f*, const~vsip\_vview\_f*);\\
\end{cfuncs}
\pyjvsiph
\viewmthd{yes}{yes}{yes}{inOut.tan}
\apyfunc{yes}{out = tan(in,out)}
\pyComment{
\item{The \ttbf{tan} function works much the same as the C VSIPL version except that a convenience pointer to the output view is returned. This may be done in-place if \ttbf{in==out}.}}
