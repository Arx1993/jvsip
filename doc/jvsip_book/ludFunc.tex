\afuncT{lud}{Lower-Upper Decomposition Class.}{generalSquareSolver}
\\\cvsiplh 
\\ \hspace*{.8cm} \vspace*{.1cm} \textbf{Available Functions }
%
%\\ \hspace*{.8cm} \vspace*{.1cm} \texttt{lud\_create}
\\ \hspace*{1.cm} {
\ttfamily\vspace{.3cm}
\begin{tabular}[H]{|l|}
\multicolumn{1}{c}{\rmfamily \bfseries Create LU Object\vspace{.1cm}}\\ \hline
vsip\_lu\_d* vsip\_lud\_create\_d(vsip\_length);\\
vsip\_lu\_f* vsip\_lud\_create\_f(vsip\_length);\\
vsip\_clu\_d* vsip\_clud\_create\_d(vsip\_length);\\
vsip\_clu\_f* vsip\_clud\_create\_f(vsip\_length);\\
\hline\end{tabular}\\}
%
%\\ \hspace*{.8cm} \vspace*{.1cm} \texttt{lud\_destroy}
\\ \hspace*{1.cm} {
\ttfamily\vspace{.3cm}
\begin{tabular}[H]{|l|}
\multicolumn{1}{c}{\rmfamily \bfseries Destroy LU Object\vspace{.1cm}}\\ \hline
int vsip\_lud\_destroy\_d(vsip\_lu\_d*);\\
int vsip\_lud\_destroy\_f(vsip\_lu\_f*);\\
int vsip\_clud\_destroy\_d(vsip\_clu\_d*);\\
int vsip\_clud\_destroy\_f(vsip\_clu\_f*);\\
\hline\end{tabular}\\}
%
%\\ \hspace*{.8cm} \vspace*{.1cm} \texttt{lud}
\\ \hspace*{1.cm}{
\ttfamily\vspace{.3cm}
\begin{tabular}[H]{|l|}
\multicolumn{1}{c}{\rmfamily \bfseries Calculate LU Decomposition\vspace{.1cm}}\\ \hline
int vsip\_lud\_d(vsip\_lu\_d*, const vsip\_mview\_d*);\\
int vsip\_lud\_f(vsip\_lu\_f*, const vsip\_mview\_f*);\\
int vsip\_clud\_d(vsip\_clu\_d*, const vsip\_cmview\_d*);\\
int vsip\_clud\_f(vsip\_clu\_f*, const vsip\_cmview\_f*);\\
\hline\end{tabular}\\}
%
%\\ \hspace*{.8cm} \vspace*{.1cm} \texttt{lusol}\\
\\ \hspace*{1.cm}{
\ttfamily\vspace{.3cm}
\begin{tabular}[H]{|l|}
\multicolumn{1}{c}{\rmfamily \bfseries Solve Using Calculated LU Decomposition\vspace{.1cm}}\\ \hline
int vsip\_lusol\_d(const vsip\_lu\_d*, vsip\_mat\_op, const vsip\_mview\_d*);\\
int vsip\_lusol\_f(const vsip\_lu\_f*, vsip\_mat\_op, const vsip\_mview\_f*);\\
int vsip\_clusol\_d(const vsip\_clu\_d*, vsip\_mat\_op, const vsip\_cmview\_d*);\\
int vsip\_clusol\_f(const vsip\_clu\_f*, vsip\_mat\_op, const vsip\_cmview\_f*);\\
\hline\end{tabular}\\}
%
%\\ \hspace*{.8cm} \vspace*{.1cm} \texttt{lud\_getattr}
\\ \hspace*{1.cm}{
\ttfamily\vspace{.3cm}
\begin{tabular}[H]{|l|}
\multicolumn{1}{c}{\rmfamily \bfseries Fill LU Attribute Structure\vspace{.1cm}}\\ \hline
void vsip\_lud\_getattr\_d(const vsip\_lu\_d*, vsip\_lu\_attr\_d*);\\
void vsip\_lud\_getattr\_f(const vsip\_lu\_f*, vsip\_lu\_attr\_f*);\\
void vsip\_clud\_getattr\_d(const vsip\_clu\_d*, vsip\_clu\_attr\_d*);\\
void vsip\_clud\_getattr\_f(const vsip\_clu\_f*, vsip\_clu\_attr\_f*);\\
\hline\end{tabular}\\}
%
\\ \pyjvsiph
\\ \hspace*{.8cm}{\textbf{View Methods\vspace{.2cm}}\\
\hspace*{1.1cm}\parbox{.9\textwidth}{
\begin{itemize}
\item {three \ttbf{view} methods have been defined for LU Decompostion.}
\subitem{\ttbf{luSolve{(xb)}} -Calculate an in-place solution to $A\vec{x}=\vec{b}$ or $A X = B$.\Bs}
\subitem{\ttbf{luInv} - A method to obtain a matrix inverse using LU for computation.}
\subitem{\ttbf{lu} - A method to obtain a computed LU object from a matrix.\Bs}
\item{\Ts Methods \ttbf{lu} and \ttbf{luInv} are defined as properties.}
\item{LU decomposition will overwrite the input matrix so use a copy to preserve the calling view.}
\end{itemize}\vspace{2mm}}}\\
%%
\hspace*{1.1cm}\textbf{Example: }\vspace*{.1cm}\\
\hspace*{1.9cm}\parbox{.85\textwidth}{\Ts Assume square matrix \ttbf{view} \ttbf{A} has been created and has data in it. Also assume we have a known matrix or vector (assumed to be a column vector) \ttbf{B} and we need to solve for unknown \ttbf{X} in $A X = B$.}\\
\hspace*{1.9cm}\parbox{.8\textwidth}{\vspace{.3cm}\hspace*{1cm}\ttbf{luObj=A.lu} \\*
will create an \ttbf{LU} decomposition object and decompose calling view A.}\\
\hspace*{1.9cm}\parbox{.8\textwidth}{\vspace{.3cm}\hspace*{1cm}\ttbf{invA=A.luInv} \\*
will return an inverse matrix for A.}\\
\hspace*{1.9cm}\parbox{.8\textwidth}{\vspace{.3cm}\hspace*{1cm}\ttbf{A.luSolve(B)} \\*
will in-place solve \ttbf{B} for \ttbf{X}}.\vspace*{.3cm}\\
%
%
\hspace*{.8cm}{\textbf{LU Class Methods\vspace*{.2cm}}\\
\hspace*{1.cm}\parbox{.9\textwidth}{To create an \ttbf{LU} object do\\
\hspace*{1.cm}\ttbf{luObj = LU(t,size)} \\
Where \ttbf{t} is a string indicating the type of \ttbf{LU} object to create and \ttbf{size} is the size of the square matrix the \ttbf{LU} object will decompose.\\
For class methods table we assume we have created an LU object we call \ttbf{luObj} and we have an input matrix \ttbf{view A} compliant with \ttbf{luObj} and some compliant \ttbf{view B} where we want to solve for the unknown \ttbf{X} in $A X = B $.\vspace{.2cm}}\\
\begin{table}
\caption{Flags and Types for LU Decomposition}
\begin{center}\begin{tabular}{|l l|}
%%
\multicolumn{2}{c}{\Ts\parbox[t]{.6\textwidth}{\center{\rmfamily \bfseries LU Decomposition Types}}}\Bs\\\hline
'lu\_d' & Real \ttbf{LU}; double precision \Bs\\\hline
'lu\_f' & Real \ttbf{LU}; float precision\Bs\\\hline
'clu\_d' & Complex \ttbf{LU}; double precision\Bs\\\hline
'clu\_f' & Complex \ttbf{LU}; float precision\Bs\\\hline
%%%
\multicolumn{2}{c}{\parbox[t]{.6\textwidth}{\center{\rmfamily \bfseries Matrix Operator Flags (\ttbf{op})}}}\Bs\\\hline
\Ts'NTRANS' or \ttbf{VSIP\_MAT\_NTRANS} & No Transpose operator\Bs\\\hline
   'TRANS' or \ttbf{VSIP\_MAT\_TRANS} & Transpose operator\Bs\\\hline
   'HERM' or \ttbf{VSIP\_MAT\_HERM} & Hermitian operator\Bs\\\hline
%%%
 \end{tabular}\end{center}\end{table}
%%
\\
\hspace*{1.cm}\parbox[t]{.9\textwidth}{\begin{tabular}{|l l|}
\multicolumn{2}{c}{\parbox[t]{.8\textwidth}{\center{\rmfamily \bfseries LU Decomposition Methods\vspace{.2cm}}}\Bs} \\ \hline
\ttbf{luObj.decompose(A)} & \parbox[t]{.6\textwidth}{Decompose matrix \ttbf{A}. The decomposition is stored in the \ttbf{luObj} but \ttbf{view A} may be overwritten so use a copy if you want to preserve \ttbf{A}\Bs}\\\hline
\ttbf{luObj.solve(op,XB)} & \parbox[t]{.6\textwidth}{Solve problem $\text{op}(A) X = B$ in-place where \ttbf{XB} is input/output \ttbf{view} and \ttbf{op} is matrix operator flag.\Bs}\\\hline
\ttbf{luObj.size} & \parbox[t]{.6\textwidth}{ Property. Returns integer length size of square matrix \ttbf{LU} object will decompose.\Bs}\\\hline
\ttbf{luObj.singular} & \parbox[t]{.6\textwidth}{Property. Returns \ttbf{True} if singular; \ttbf{False} if inverse exists.\Bs}\\\hline
\ttbf{luObj.type} & \parbox[t]{.6\textwidth}{Returns string indicating LU type.\Bs}\\\hline
\ttbf{luObj.vsip} & \parbox[t]{.6\textwidth}{Returns C VSIPL LU instance.\vspace*{.1cm}\Bs}\\\hline
\end{tabular}\vspace*{.4cm}}\\
