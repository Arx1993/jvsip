\afuncT{prodt}{Matrix transpose product.}{matrixOperations}
\\\cvsiplh
\\ \hspace*{.8cm} \vspace*{.1cm} \textbf{Available Functions }
\\ \hspace*{0.03\textwidth} {
\ttfamily
\begin{tabular}[H]{l}
void vsip\_cmprodt\_d(\\*\hspace{.6cm}
    const vsip\_cmview\_d*, const vsip\_cmview\_d*, const vsip\_cmview\_d*);\\
void vsip\_cmprodt\_f(\\*\hspace{.6cm}
    const vsip\_cmview\_f*, const vsip\_cmview\_f*, const vsip\_cmview\_f*);\\
void vsip\_mprodt\_d(\\*\hspace{.6cm}
    const vsip\_mview\_d*, const vsip\_mview\_d*, const vsip\_mview\_d*);\\
void vsip\_mprodt\_f(\\*\hspace{.6cm}
    const vsip\_mview\_f*, const vsip\_mview\_f*, const vsip\_mview\_f*);\\
\end{tabular}
}
\\\pyjvsiph
\viewmthd{yes}{No}{No}{out=inOne.prodt(inTwo)}
\apyfunc{yes}{out = prodt(inOne,inTwo,out)}
\pyComment{
\item{The \ttbf{prodt} function works much the same as the C VSIPL version except that a convenience pointer to the output view is returned. This may not be done in-place.}
\item{The \ttbf{prodt} method creates and returns a new \ttbf{view} with the result.}
}