\afuncT{atan}{Computes the principal radian value, $[-\pi/2,\pi/2]$, of the arctangent for each element of a \ttbf{view}.}{elementaryMath}
\\\cvsiplh 
\begin{cfuncs}
vsip\_scalar\_f vsip\_atan\_f(vsip\_scalar\_f);\\
vsip\_scalar\_d vsip\_atan\_d(vsip\_scalar\_d);\\
void vsip\_matan\_d(const~vsip\_mview\_d*, const~vsip\_mview\_d*);\\
void vsip\_matan\_f(const~vsip\_mview\_f*, const~vsip\_mview\_f*);\\
void vsip\_vatan\_d(const~vsip\_vview\_d*, const~vsip\_vview\_d*);\\
void vsip\_vatan\_f(const~vsip\_vview\_f*, const~vsip\_vview\_f*);\\
\end{cfuncs}
\pyjvsiph
\viewmthd{yes}{yes}{yes}{inOut.atan}
\apyfunc{yes}{out = atan(in,out)}
\begin{comments}
\item{The \ttbf{atan} function works much the same as the C VSIPL version except that a convenience pointer to the output view is returned. 
\item The \ttbf{atan} function may be done in-place if \ttbf{in==out}.}
\end{comments}