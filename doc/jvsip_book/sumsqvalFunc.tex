\afuncT{sumsqval}{Returns the sum of the squares of all the elements of a \ttbf{view}. Does not modify input.}{unaryOperations}
\\\cvsiplh
\afh
\\\hspace*{.04\textwidth} {
\ttfamily
\begin{tabular}[H]{l}
vsip\_scalar\_d vsip\_msumsqval\_d(const vsip\_mview\_d* );\\
vsip\_scalar\_d vsip\_vsumsqval\_d(const vsip\_vview\_d* );\\
vsip\_scalar\_f vsip\_msumsqval\_f(const vsip\_mview\_f* );\\
vsip\_scalar\_f vsip\_vsumsqval\_f(const vsip\_vview\_f* );\\
\end{tabular}
}
\\\pyjvsiph
\viewmthd{yes}{yes}{NA}{aValue=in.sumsqval}
\apyfunc{No}{}
\\ \hspace*{.8cm} \textbf{Comments}
\\\hspace*{.9cm}\parbox{10.8cm}{\vspace*{.1cm}\begin{itemize}
\item{Since the \ttbf{sumsqval} function returns a scalar without modifying the \ttbf{view} there seemed little point in supporting this as a separate function call for \pyjv.}
\end{itemize}
}
