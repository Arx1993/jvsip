\clearpage
{\large \textbf{\hypertarget{fftFunc}{FFT Function Set}}}\vspace{.2cm}\\
\hspace*{.3cm}
\parbox{0.85\textwidth}{Discrete Fourier Transforms. See FFT Functions table \ref{tab:fftFunctions}}
\cvsiplh 
\newline \hspace*{.8cm} \vspace*{.1cm} \textbf{Available Functions }
\newline \hspace*{.8cm} \vspace*{.1cm} \texttt{fft\_create}
\newline \hspace*{1.1cm} {
\ttfamily
\begin{tabular}[H]{l}\hline
\hline \multicolumn{1}{c}{\rmfamily \bfseries FFT Create Functions\vspace{.1cm}}\\ \hline
vsip\_fft\_d* vsip\_ccfftip\_create\_d(vsip\_length, \\*\hspace{.7cm}vsip\_scalar\_d,vsip\_fft\_dir,\\*\hspace{.7cm} unsigned int, vsip\_alg\_hint);\vspace{.1cm}\\
vsip\_fft\_d* vsip\_ccfftop\_create\_d(vsip\_length, \\*\hspace{.7cm}vsip\_scalar\_d,vsip\_fft\_dir,\\*\hspace{.7cm} unsigned int, vsip\_alg\_hint);\vspace{.1cm}\\
vsip\_fft\_d* vsip\_crfftop\_create\_d(vsip\_length,\\*\hspace{.7cm}vsip\_scalar\_d,\\*\hspace{.7cm}unsigned int, vsip\_alg\_hint);\vspace{.1cm}\\
vsip\_fft\_d* vsip\_rcfftop\_create\_d(vsip\_length,\\*\hspace{.7cm}vsip\_scalar\_d,\\*\hspace{.7cm}unsigned int, vsip\_alg\_hint);\vspace{.1cm}\\
vsip\_fft\_f* vsip\_ccfftip\_create\_f(vsip\_length,\\*\hspace{.7cm}vsip\_scalar\_f,vsip\_fft\_dir,\\*\hspace{.7cm}unsigned int, vsip\_alg\_hint);\vspace{.1cm}\\
vsip\_fft\_f* vsip\_ccfftop\_create\_f(vsip\_length,\\*\hspace{.7cm}vsip\_scalar\_f,vsip\_fft\_dir,\\*\hspace{.7cm}unsigned int, vsip\_alg\_hint);\vspace{.1cm}\\
vsip\_fft\_f* vsip\_crfftop\_create\_f(vsip\_length,\\*\hspace{.7cm}vsip\_scalar\_f,\\*\hspace{.7cm}unsigned int, vsip\_alg\_hint);\vspace{.1cm}\\
vsip\_fft\_f* vsip\_rcfftop\_create\_f(vsip\_length,\\*\hspace{.7cm}vsip\_scalar\_f,\\*\hspace{.7cm}unsigned int, vsip\_alg\_hint);\vspace{.1cm}\\
\end{tabular}
}
\newline\hspace*{1.1cm} {
\begin{tabular}[H]{l}\hline
\multicolumn{1}{c}{\rmfamily \bfseries Multiple FFT Create Functions}\\ \hline
vsip\_fftm\_d* vsip\_ccfftmip\_create\_d(\\*\hspace{.7cm}vsip\_length, vsip\_length,\\*\hspace{.7cm}vsip\_scalar\_d, vsip\_fft\_dir, vsip\_major,\\*\hspace{.7cm} unsigned int,vsip\_alg\_hint);\\
vsip\_fftm\_d* vsip\_ccfftmop\_create\_d(\\*\hspace{.7cm}vsip\_length, vsip\_length,\\*\hspace{.7cm}vsip\_scalar\_d, vsip\_fft\_dir, vsip\_major,\\*\hspace{.7cm} unsigned int,vsip\_alg\_hint);\\
vsip\_fftm\_d* vsip\_crfftmop\_create\_d(\\*\hspace{.7cm}vsip\_length, vsip\_length,\\*\hspace{.7cm}vsip\_scalar\_d, vsip\_major,\\*\hspace{.7cm} unsigned int, vsip\_alg\_hint);\\
vsip\_fftm\_d* vsip\_rcfftmop\_create\_d(\\*\hspace{.7cm}vsip\_length, vsip\_length,\\*\hspace{.7cm}vsip\_scalar\_d, vsip\_major,\\*\hspace{.7cm} unsigned int, vsip\_alg\_hint);\\
vsip\_fftm\_f* vsip\_ccfftmip\_create\_f(\\*\hspace{.7cm}vsip\_length, vsip\_length,\\*\hspace{.7cm}vsip\_scalar\_f, vsip\_fft\_dir, vsip\_major,\\*\hspace{.7cm} unsigned int,vsip\_alg\_hint);\\
vsip\_fftm\_f* vsip\_ccfftmop\_create\_f(\\*\hspace{.7cm}vsip\_length, vsip\_length,\\*\hspace{.7cm}vsip\_scalar\_f, vsip\_fft\_dir, vsip\_major,\\*\hspace{.7cm} unsigned int,vsip\_alg\_hint);\\
vsip\_fftm\_f* vsip\_crfftmop\_create\_f(\\*\hspace{.7cm}vsip\_length, vsip\_length,\\*\hspace{.7cm}vsip\_scalar\_f, vsip\_major,\\*\hspace{.7cm} unsigned int, vsip\_alg\_hint);\\
vsip\_fftm\_f* vsip\_rcfftmop\_create\_f(\\*\hspace{.7cm}vsip\_length, vsip\_length,\\*\hspace{.7cm}vsip\_scalar\_f, vsip\_major,\\*\hspace{.7cm} unsigned int, vsip\_alg\_hint);\\ \hline
\end{tabular}
}
\clearpage
\hspace*{.8cm} \vspace*{.1cm} \texttt{fft\_destroy}
\newline \hspace*{1.1cm} {
\ttfamily
\begin{tabular}[H]{l}\hline
\hline \multicolumn{1}{c}{\rmfamily \bfseries FFT Destroy Functions}\\ \hline
int vsip\_fft\_destroy\_d(vsip\_fft\_d*);\\
int vsip\_fft\_destroy\_f(vsip\_fft\_f*);\\
\hline \multicolumn{1}{c}{\rmfamily \bfseries Multiple FFT Destroy Functions}\\ \hline
int vsip\_fftm\_destroy\_d(vsip\_fftm\_d*);\\
int vsip\_fftm\_destroy\_f(vsip\_fftm\_f*);\\ \hline
\end{tabular}
}\vspace{.1cm}
\newline \hspace*{.8cm} \vspace*{.1cm} \texttt{fft}
\newline \hspace*{1.1cm} {
\ttfamily
\begin{tabular}[H]{l} \hline
\hline \multicolumn{1}{c}{\rmfamily \bfseries FFT Functions}\\ \hline
void vsip\_ccfftip\_d(const vsip\_fft\_d*,\\*\hspace{.7cm}const vsip\_cvview\_d*);\\
void vsip\_ccfftip\_f(const vsip\_fft\_f*,\\*\hspace{.7cm} const vsip\_cvview\_f*);\\
void vsip\_ccfftop\_d(const vsip\_fft\_d*,\\*\hspace{.7cm} const vsip\_cvview\_d*, const vsip\_cvview\_d*);\\
void vsip\_ccfftop\_f(const vsip\_fft\_f*,\\*\hspace{.7cm} const vsip\_cvview\_f*, const vsip\_cvview\_f*);\\
void vsip\_crfftop\_d(const vsip\_fft\_d*,\\*\hspace{.7cm} const vsip\_cvview\_d*, const vsip\_vview\_d*);\\
void vsip\_crfftop\_f(const vsip\_fft\_f*,\\*\hspace{.7cm} const vsip\_cvview\_f*, const vsip\_vview\_f*);\\
void vsip\_rcfftop\_d(const vsip\_fft\_d*,\\*\hspace{.7cm} const vsip\_vview\_d*, const vsip\_cvview\_d*);\\
void vsip\_rcfftop\_f(const vsip\_fft\_f*,\\*\hspace{.7cm} const vsip\_vview\_f*, const vsip\_cvview\_f*);\\ \hline \multicolumn{1}{c}{\rmfamily \bfseries Multiple FFT Functions}\\ \hline
void vsip\_ccfftmip\_d(const vsip\_fftm\_d*,\\*\hspace{.7cm} const vsip\_cmview\_d*);\\
void vsip\_ccfftmip\_f(const vsip\_fftm\_f*,\\*\hspace{.7cm} const vsip\_cmview\_f*);\\
void vsip\_ccfftmop\_d(const vsip\_fftm\_d*,\\*\hspace{.7cm} const vsip\_cmview\_d*, const vsip\_cmview\_d*);\\
void vsip\_ccfftmop\_f(const vsip\_fftm\_f*,\\*\hspace{.7cm} const vsip\_cmview\_f*, const vsip\_cmview\_f*);\\
void vsip\_crfftmop\_d(const vsip\_fftm\_d*,\\*\hspace{.7cm} const vsip\_cmview\_d*, const vsip\_mview\_d*);\\
void vsip\_crfftmop\_f(const vsip\_fftm\_f*,\\*\hspace{.7cm} const vsip\_cmview\_f*, const vsip\_mview\_f*);\\
void vsip\_rcfftmop\_d(const vsip\_fftm\_d*,\\*\hspace{.7cm} const vsip\_mview\_d*, const vsip\_cmview\_d*);\\
void vsip\_rcfftmop\_f(const vsip\_fftm\_f*,\\*\hspace{.7cm} const vsip\_mview\_f*, const vsip\_cmview\_f*);\\ \hline \end{tabular}
}
\clearpage
\hspace*{.8cm} \texttt{fft\_getattr}
\newline \hspace*{1.1cm} {
\ttfamily
\begin{tabular}[H]{l}\hline
\hline \multicolumn{1}{c}{\rmfamily \bfseries FFT Get Attributes Functions}\\ \hline
void vsip\_fft\_getattr\_d(\\*\hspace{.7cm}const vsip\_fft\_d*, vsip\_fft\_attr\_d*);\\
void vsip\_fft\_getattr\_f(\\*\hspace{.7cm}const vsip\_fft\_f*, vsip\_fft\_attr\_f*);\\
\hline \multicolumn{1}{c}{\rmfamily \bfseries Multiple FFT Get Attributes Functions}\\ \hline
void vsip\_fftm\_getattr\_d(\\*\hspace{.7cm}const vsip\_fftm\_d*, vsip\_fftm\_attr\_d*);\\
void vsip\_fftm\_getattr\_f(\\*\hspace{.7cm}const vsip\_fftm\_f*, vsip\_fftm\_attr\_f*);\\
\end{tabular}
}
\pyjvsiph
\newline\hspace*{.8cm}{\textbf{View Methods\vspace{.2cm}}\\
\hspace*{1cm}\parbox{10.5cm}{
\begin{itemize}
\item {Special \ttbf{view} methods exist for \ttbf{view}s of type float and double.} 
\item {\ttbf{View} methods are defined as properties (@property) so the scale factor is always one for \ttbf{view} methods.}
\item {For out of place the method will create and return the output \ttbf{view}.
\item {\ttbf{View} methods determine if the FFT is a multiple FFT or a vector FFT by the \ttbf{view} type. The \pyjv \ttbf{view} \ttbf{major} attribute is used to determine if the multiple FFT is by row or by column.}
\item {Out-of-place \ttbf{view} methods \ttbf{fftop} and \ttbf{ifftop} treat real vectors as if they were complex with a zero imaginary part}
}
\end{itemize}}\\
\hspace*{1.1cm}\textbf{In-Place: }\hspace{.2cm} yes\\
\hspace*{1.1cm}\textbf{Example: }\\
\hspace*{2.9cm}Forward transform of vector \ttbf{x} in-place\\*
\hspace*{3.5cm}\ttbf{x.fftip}\\
\hspace*{2.9cm}For matrix FFT multiple use major attribute.\\*
\hspace*{3.5cm}\ttbf{x.ROW.fftip} \\*
\hspace*{3.5cm}\ttbf{x.COL.fftip}\\
\hspace*{2.9cm}Inverse transform of vector \ttbf{x} in-place\\*
\hspace*{3.5cm}\ttbf{x.ifftip}\\
\hspace*{1.1cm}\textbf{Out-of-Place: }\hspace{.2cm} yes\\
\hspace*{1.1cm}\textbf{Example: }\\
\hspace*{2.9cm}Real to complex and complex to real FFT\\*
\hspace*{3.5cm}\ttbf{ y=x.rcfft}\\*
\hspace*{3.5cm}\ttbf{ z=y.crfft}\\*
\hspace*{2.9cm}Complex to complex transform of vector \ttbf{x} out-of-place\\
\hspace*{3.5cm}\ttbf{ y=x.fftop}\\
 \hspace*{2.9cm}Complex to complex inverse transform of vector \ttbf{x}\\*\hspace*{2.9cm}out-of-place\\
\hspace*{3.5cm}\ttbf{ y=x.ifftop}\\
 \hspace*{2.9cm}Complex to complex multiple transform of matrix \ttbf{x}\\*\hspace*{2.9cm}out-of-place by column\\
\hspace*{3.5cm}\ttbf{ y=x.COL.fftop}\\
 \hspace*{2.9cm}Complex to complex multiple transform of matrix \ttbf{x}\\*\hspace*{2.9cm}out-of-place by row\\
\hspace*{3.5cm}\ttbf{ y=x.ROW.fftop}\\
 \hspace*{2.9cm}Complex to complex multiple inverse transform of matrix \ttbf{x}\\*\hspace*{2.9cm}out-of-place by column\\
\hspace*{3.5cm}\ttbf{ y=x.COL.ifftop}\\
 \hspace*{2.9cm}Complex to complex multiple inverse transform of matrix \ttbf{x}\\*\hspace*{2.9cm}out-of-place by row\\
\hspace*{3.5cm}\ttbf{ y=x.ROW.ifftop}\\
\hspace*{.8cm}{\textbf{FFT Class\vspace{.2cm}}\\
\hspace*{1.cm}\parbox{.9\textwidth}{To create an FFT object use \\*
\hspace*{1.cm} \ttbf{fftObj=FFT(t,*args)}\\*
where \ttbf{args} is a tuple containing the create parameters for the FFT type selected, and \ttbf{t} is a string indicating the type of FFT to create.\vspace{.2cm}}\\
\hspace*{1.cm}\parbox{.9\textwidth}{Note \ttbf{args} will contain some or all of the following in the order listed. The exact argument list, and each type string, is shown in the Discrete Fourier Transform Class table below.\\
\begin{tabular}[t]{|l l|}\hline
\ttbf{M} & \parbox[t]{.75\textwidth}{Column Length \vspace*{.1cm}}\\ \hline
\ttbf{N} & \parbox[t]{.75\textwidth}{Row Length for \ttbf{matrix} or Vector length for \ttbf{vector}\vspace*{.1cm}} \\\hline
\ttbf{scl} & \parbox[t]{.75\textwidth}{Scale Factor \vspace*{.1cm}}\\\hline
\ttbf{dir} & \parbox[t]{.75\textwidth}{Direction flag for FFT either VSIP\_FFT\_FWD or VSIP\_FFT\_INV \vspace*{.1cm}}\\\hline
\ttbf{major} & \parbox[t]{.75\textwidth}{For multiple FFT by row (VSIP\_ROW) or by column (VSIP\_COL) \vspace*{.1cm}}\\\hline
\ttbf{ntimes} & \parbox[t]{.75\textwidth}{Hint for how much the FFT object will be used. Zero indicates many times\vspace*{.1cm}} \\\hline
\ttbf{alghint} & \parbox[t]{.75\textwidth}{Algorithm hint to optimize for speed (VSIP\_ALG\_TIME), size (VSIP\_ALG\_SPACE), or accuracy (VSIP\_ALG\_NOISE)\vspace*{.1cm}}\\
\hline \end{tabular}}
\newline
\hspace*{1.cm}\parbox[t]{.85\textwidth}{\begin{tabular}{|l l|}\hline
\multicolumn{2}{|c|}{\parbox[t]{.68\textwidth}{\center{\rmfamily \bfseries Discrete Fourier Transform Class}\vspace{.2cm}}}\\ \hline \hline
'ccfftip\_f' & \parbox[t]{.68\textwidth}{Complex-to-complex FFT float precision in-place \\*\ttbf{args = (M,N,scl,dir,ntimes,alghint)}\vspace*{.1cm}}\\\hline
'ccfftop\_f' & \parbox[t]{.68\textwidth}{Complex-to-complex FFT float precision out-of-place \\*\ttbf{args = (M,N,scl,dir,ntimes,alghint)}\vspace*{.1cm}}\\\hline
'rcfftop\_f' & \parbox[t]{.68\textwidth}{Real-to-complex FFT float precision out-of-place \\*\ttbf{args = (M,N,scl,ntimes,alghint)}\vspace*{.1cm}}\\\hline
'crfftop\_f'& \parbox[t]{.68\textwidth}{Complex-to-real FFT single precision out-of-place\\*\ttbf{args = (M,N,scl,ntimes,alghint)}\vspace*{.1cm}}\\\hline
'ccfftip\_d' & \parbox[t]{.68\textwidth}{Complex-to-complex FFT double precision in-place\\*\ttbf{args = (M,N,scl,dir,ntimes,alghint)}\vspace*{.1cm}}\\\hline
'ccfftop\_d'& \parbox[t]{.68\textwidth}{Complex-to-complex FFT double precision out-of-place\\*\ttbf{args = (M,N,scl,dir,ntimes,alghint)}\vspace*{.1cm}}\\\hline
'rcfftop\_d'& \parbox[t]{.68\textwidth}{Real-to-complex multiple FFT single precision out-of-place\\*\ttbf{args = (M,N,scl,ntimes,alghint)}\vspace*{.1cm}}\\\hline
'crfftop\_d'& \parbox[t]{.68\textwidth}{Complex-to-real multiple FFT single precision out-of-place\\*\ttbf{args = (M,N,scl,ntimes,alghint)}\vspace*{.1cm}}\\\hline
'ccfftmip\_f' & \parbox[t]{.68\textwidth}{Complex-to-complex multiple FFT single precision in-place\\*\ttbf{args = (M,N,scl,dir,major,ntimes,alghint)}\vspace*{.1cm}}\\\hline
'ccfftmop\_f' & \parbox[t]{.68\textwidth}{Complex-to-complex multiple FFT single precision out-of-place\\*\ttbf{args = (M,N,scl,dir,major,ntimes,alghint)}\vspace*{.1cm}}\\\hline
'rcfftmop\_f' & \parbox[t]{.68\textwidth}{Real-to-complex multiple FFT single precision out-of-place\\*\ttbf{args = (M,N,scl,major,ntimes,alghint)}\vspace*{.1cm}}\\\hline
'crfftmop\_f' &\parbox[t]{.68\textwidth}{ Complex-to-real multiple FFT single precision out-of-place\\*\ttbf{args = (M,N,major,ntimes,alghint)}\vspace*{.1cm}}\\\hline
'ccfftmip\_d' & \parbox[t]{.68\textwidth}{Complex-to-complex multiple FFT double precision in-place\\*\ttbf{args =(M,N,scl,dir,major,ntimes,alghint)}\vspace*{.1cm}}\\\hline
'ccfftmop\_d' & \parbox[t]{.68\textwidth}{Complex-to-complex multiple FFT double precision out-of-place\\*\ttbf{args = (M,N,scl,dir,major,ntimes,alghint)}\vspace*{.1cm}}\\\hline
'rcfftmop\_d' & \parbox[t]{.68\textwidth}{Real-to-complex multiple FFT double precision out-of-place\\*\ttbf{args = (M,N,scl,major,ntimes,alghint)}\vspace*{.1cm}}\\\hline
'crfftmop\_d' & \parbox[t]{.68\textwidth}{Complex-to-real multiple FFT double precision out-of-place\\*\ttbf{args = (M,N,scl,major,ntimes,alghint)}\vspace*{.1cm}}\\
\hline\end{tabular}}
\clearpage\hspace*{.8cm}{\textbf{FFT Class Methods}\\
\hspace*{1.1cm} \parbox[t]{.88\textwidth}{Below we assume we have created an FFT object we call \ttbf{fftObj} and we have an input \ttbf{view x} compliant with \ttbf{fftObj} and if necessary a compliant output \ttbf{view y}.\vspace{.2cm}}
\newline\hspace*{1.2cm}\parbox[t]{.85\textwidth}{To calculate an in-place DFT we do\\*\hspace*{.5cm}\ttbf{fftObj.dft(x)}\\ To calculate an out-of-place DFT we do\\*\hspace*{.5cm} \ttbf{fftObj.dft(x,y)\vspace{.1cm}}\\
To get the FFT type (a string) we do\\*\hspace*{.5cm}\ttbf{t=fftObj.type}\vspace{.1cm}\\To get the argument list (a tuple) the FFT was created with we do\\*\hspace*{.5cm}\ttbf{arg=fftObj.arg}\vspace{.1cm}\\If we want to examine or use the C VSIPL FFT Object encapsulated inside the pyJvsip FFT object we do\\*\hspace*{.5cm}\ttbf{vsipObj=fftObj.vsip\\}}
