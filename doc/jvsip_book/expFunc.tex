\afuncT{exp}{Exponential; An elementwise function}{elementaryMath}
\\\cvsiplh
\afh
\\\hspace*{.04\textwidth} {
\ttfamily
\begin{tabular}[H]{l}
vsip\_cscalar\_d vsip\_cexp\_d(vsip\_cscalar\_d);\\
vsip\_cscalar\_f vsip\_cexp\_f(vsip\_cscalar\_f);\\
void vsip\_cmexp\_d(const vsip\_cmview\_d*, const vsip\_cmview\_d*);\\
void vsip\_cmexp\_f(const vsip\_cmview\_f*, const vsip\_cmview\_f*);\\
void vsip\_cvexp\_d(const vsip\_cvview\_d*, const vsip\_cvview\_d*);\\
void vsip\_cvexp\_f(const vsip\_cvview\_f*, const vsip\_cvview\_f*);\\
void vsip\_mexp\_d(const vsip\_mview\_d*, const vsip\_mview\_d*);\\
void vsip\_mexp\_f(const vsip\_mview\_f*, const vsip\_mview\_f*);\\
void vsip\_vexp\_d(const vsip\_vview\_d*, const vsip\_vview\_d*);\\
void vsip\_vexp\_f(const vsip\_vview\_f*, const vsip\_vview\_f*);\\
\end{tabular}
}
\\\pyjvsiph
\viewmthd{yes}{yes}{yes}{inOut.exp}
\apyfunc{yes}{out = exp(in,out)}
\pyComment{
\item{The \ttbf{exp} function works much the same as the C VSIPL version except that a convenience pointer to the output view is returned. This may be done in-place if \ttbf{in==out}.}}
