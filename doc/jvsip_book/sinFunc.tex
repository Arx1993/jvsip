\afunc{sin}{Sine; An element-wise function. Input \ttbf{view} elements are assumed to be in radians. See elementary math functions table \ref{tab:elementaryMath}.}
\cvsiplh
\newline \hspace*{.8cm} \vspace*{.1cm} \textbf{Available Functions }
\newline \hspace*{1.1cm} {
\ttfamily
\begin{tabular}[H]{l}
vsip\_scalar\_f vsip\_sin\_f(vsip\_scalar\_f a);\\
vsip\_scalar\_d vsip\_sin\_d(vsip\_scalar\_d a);\\
void vsip\_msin\_d(\\*
\hspace{1cm}const vsip\_mview\_d*, const vsip\_mview\_d*);\\
void vsip\_msin\_f(\\*
\hspace{1cm}const vsip\_mview\_f*, const vsip\_mview\_f*);\\
void vsip\_vsin\_d(\\*
\hspace{1cm}const vsip\_vview\_d*, const vsip\_vview\_d*);\\
void vsip\_vsin\_f(\\*
\hspace{1cm}const vsip\_vview\_f*, const vsip\_vview\_f*);\\
\end{tabular}
}
\pyjvsiph
\viewmthd{yes}{yes}{yes}{inOut.sin}
\apyfunc{yes}{out = sin(in,out)}
\newline\hspace*{1.2cm}\parbox{10.8cm}{\vspace*{.1cm}The \ttbf{sin} function works much the same as the C VSIPL version except that a convenience pointer to the output view is returned. This may be done in-place if \ttbf{in==out}.}
