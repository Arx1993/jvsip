\afuncT{sinh}{Hyperbolic Sine; An elementwise function. }{elementaryMath}
\\\cvsiplh
\afh
\\\hspace*{.04\textwidth} {
\ttfamily
\begin{tabular}[H]{l}
vsip\_scalar\_f vsip\_sinh\_f(vsip\_scalar\_f a);\\
vsip\_scalar\_d vsip\_sinh\_d(vsip\_scalar\_d a);\\
void vsip\_msinh\_d(const vsip\_mview\_d*, const vsip\_mview\_d*);\\
void vsip\_msinh\_f(const vsip\_mview\_f*, const vsip\_mview\_f*);\\
void vsip\_vsinh\_d(const vsip\_vview\_d*, const vsip\_vview\_d*);\\
void vsip\_vsinh\_f(const vsip\_vview\_f*, const vsip\_vview\_f*);\\
\end{tabular}
}
\\\pyjvsiph
\viewmthd{yes}{yes}{yes}{inOut.sinh}
\apyfunc{yes}{out = sinh(in,out)}
\pyComment{
\item{The \ttbf{sinh} function works much the same as the C VSIPL version except that a convenience pointer to the output view is returned. This may be done in-place if \ttbf{in==out}.}}
