\afunc{modulate}{Computes the modulation of a real vector by a specified complex frequency. See unary operations table \ref{tab:unaryOperations}}
\\\cvsiplh
\newline \hspace*{.8cm} \vspace*{.1cm} \textbf{Available Functions }
\newline \hspace*{1.1cm} {
\ttfamily
\begin{tabular}[H]{l}
vsip\_scalar\_d vsip\_cvmodulate\_d(\\*\hspace{.6cm}const vsip\_cvview\_d*, vsip\_scalar\_d, \\*\hspace{.6cm}vsip\_scalar\_d, const vsip\_cvview\_d*);\\
vsip\_scalar\_d vsip\_vmodulate\_d(\\*\hspace{.6cm}const vsip\_vview\_d*, vsip\_scalar\_d, \\*\hspace{.6cm}vsip\_scalar\_d, const vsip\_cvview\_d*);\\
vsip\_scalar\_f vsip\_cvmodulate\_f(\\*\hspace{.6cm}const vsip\_cvview\_f*, vsip\_scalar\_f, \\*\hspace{.6cm}vsip\_scalar\_f, const vsip\_cvview\_f*);\\
vsip\_scalar\_f vsip\_vmodulate\_f(\\*\hspace{.6cm}const vsip\_vview\_f*, vsip\_scalar\_f, \\*\hspace{.6cm}vsip\_scalar\_f, const vsip\_cvview\_f*);\\
\end{tabular}
}
\\\pyjvsiph
\viewmthd{No}{NA}{NA}{NA}
\apyfunc{Yes}{phiNew,out=modulate(in,nu,phi,out)}
\pyComment{{\item{Note \ttbf{phi} is the initial phase and the final phase is returned as \ttbf{phiNew}. For \pyjv we also return a convenience copy of the output vector}}}