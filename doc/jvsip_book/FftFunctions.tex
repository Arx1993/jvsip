\subsubsection*{Fast Fourier Transforms} \addcontentsline{toc}{subsubsection}{Fast Fourier Transforms}
Discrete Fourier transforms are done using an FFT algorithm.  Although the VSIPL 1.3 specification has definitions for two and three dimensional FFTs \jv only supports the one dimensional version.
\begin{table}[H]
\caption{FFT Functions \ref{tab:signalProcessingFunctions}}
\label{tab:fftFunctions}
\begin{center}
\begin{tabular}{|l|l|}
\multicolumn{2}{c}{\rmfamily \bfseries Discrete Fourier Transform Class}\\
\multicolumn{2}{c}{See function page \hyperlink{fftFunc}{\texttt{FFT}}} \\ \hline
fft & Execute FFT\\
fft\_create &Create FFT Object\\
fft\_setwindow &Set a window in the FFT object\\
fft\_destroy & Free FFT object\\
fft\_getattr & Get attributes of FFT object\\
\hline\end{tabular}
\end{center}
\label{default}
\end{table}%