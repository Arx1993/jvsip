\chapter{Signal Processing}
\section*{Introduction}
\newpage
\cex
\inputminted[linenos=true,resetmargins=true]{c}{./c_examples/example18.c}
\newpage
\section*{Fourier Transforms}\addcontentsline{toc}{section}{Fourier Transforms}
\subsection*{Vector FFT}\addcontentsline{toc}{subsection}{Vector FFT}
\subsection*(Vector FFT by Row or Column}\addcontentsline{toc}{subsection}{Vector FFT by Row or Column}
\section*{Convolution, Correlation and FIR Filtering}\addcontentsline{toc}{section}{Convolution, Correlation and FIR Filtering}
\section*{Window Creation}\addcontentsline{toc}{section}{Window Creation}
{VSIPL provides functions to create Blackman, Chebyshev, Hanning and Kaiser windows. 
Unlike most functions in C VSIPL the window creation routines do not use an already 
created vector and fill it. Instead they actually create a block, allocate data for the 
block, create a unit stride full length vector on the block, fill the vector with the window 
coefficients, and then return the pointer to the vector view. The return value will be 
\ilCode{NULL} on an allocation failure.

For pyJvsip the windows are defined as a method on a view so the functionality, from a 
user perspective, is to create a vector of a certain type and length and then fill the 
vector with a window.  Size information are taken from the calling view. Under the covers 
the C VSIPL window functions are used so a copy is actually taking place to meet the 
requirements of pyJvsip.}

\section*{Miscellaneous}\addcontentsline{toc}{section}{Miscellaneous}
\subsection*{Histogram}\addcontentsline{toc}{subsection}{Histogram}
\subsection*{Data Reorganization}\addcontentsline{toc}{subsection}{Data Reorganization}
\subsubsection*{Frequency Swapping}\addcontentsline{toc}{subsubsection}{Frequency Swapping}
